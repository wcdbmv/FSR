\documentclass[12pt, a4paper]{article}

\include{preamble}

\begin{document}


\noindent УДК~004.94

\hfill

\noindent \textbf{Сравнительный анализ разностного и вероятностного методов исследования математической модели, построенной на дифференциальном уравнении в частных производных эллиптического типа}

\noindent А.~Ш.~Керимов$^{1}$\hfill kerimov.edu@yandex.ru

\noindent $^{1}$МГТУ им.~Н.~Э.~Баумана, Москва, Россия

\hfill

\noindent \textbf{Аннотация}

\noindent Статья посвящена численным методам приближённого решения стационарного многомерного эллиптического уравнения теплопроводности.
Выведены математические соотношения двух конечно-разностных алгоритмов: метода Либмана для центрально-симметричной схемы и метода прогонки для продольно-поперечной схемы; а также вероятностного метода. Проведён анализ программных реализаций методов на быстродействие.

\noindent \textbf{Ключевые слова}

\noindent \textit{задача двумерной теплопроводности, численные методы, уравнение Пуассона, продольно-поперечная схема, матричная прогонка, центрально-симметричная схема, метод Либмана, вероятностный метод}

\hfill

\section*{Введение}

Прикладные проблемы приводят к необходимости решения краевых задач для уравнений с частными производными. Разработка приближённых методов их решения базируется на построении и исследовании численных методов решения краевых задач для базовых (основных, модельных) уравнений математической физики. В качестве таковых при рассмотрении уравнений второго порядка выделяются эллиптические, параболические и гиперболические уравнения.

\section{Постановка задачи}

Задана математическая модель с постоянными коэффициентами $k(x, z) \equiv k$ (уравнение Пуассона):

\begin{equation}
	\label{eqn:source}
	\frac{\partial^2 u}{\partial x^2} + \frac{\partial^2 u}{\partial z^2} + \frac{f(x, z)}{k} = 0.
\end{equation}

На границах прямоугольной области $(0 < x < a, 0 < z < b)$.

\begin{center}
	\scalebox{0.8}{
	\begin{tikzpicture}
		\begin{axis}[
			xlabel={$x$},
			ylabel={$z$},
			every axis x label/.style={at={(current axis.right of origin)},anchor=west},
			every axis y label/.style={at={(current axis.north west)},above=2mm},
			axis lines=left,
			xmin=0,
			xmax=1.1,
			ymin=0,
			ymax=1.25,
			xtick={0, 1},
			xticklabels={$0$, $a$},
			ytick={1},
			yticklabels={$b$},
			tick style={draw=none},
			]
			\addplot[
			mark=none,
			very thick,
			] coordinates{
				(0, 1)
				(1, 1)
			};

			\addplot[
			mark=none,
			very thick,
			] coordinates {
				(1, 0)
				(1, 1)
			};
		\end{axis}
	\end{tikzpicture}
}
\end{center}

Краевые условия:

\begin{equation}
	\begin{dcases}
		x = 0, & u(0, z) = u_0, \\
		x = a, & u(a, z) = u_0, \\
		z = 0, & u(x, 0) = u_0, \\
		z = b, & u(x, b) = u_0. \\
	\end{dcases}
\end{equation}

Значения коэффициентов задачи (все размерности согласованы).

Геометрические размеры $a = b = 10$ см.

$u_0 = 300$ К.

В качестве примера функции источников можно предложить распределение вида $f(x, z) = f_0 e^{\beta(x - a/2)^2(z - b/2)^2}$, параметры $f_0, \beta$ варьируются исходя из условия, чтобы максимум функции не превышал $3000$~K.

Сформулированная математическая модель описывает двумерное температурное поле $u(x, z)$ в тонкой прямоугольной пластине с размерами $a \times b$.
Температура по толщине пластины (третьей координате) принимается постоянной.

\section{Конечно-разностная аппроксимация}

\subsection{Разностно-итерационный метод Либмана}

Наложим сетку

\begin{equation}
	\omega_{h_x, h_z} = \{x_i = ih_x, i = \overline{0,N_x}; \;\; z_j = jh_z, j = \overline{0, N_z}\}.
\end{equation}

На этой сетке аппроксимируем дифференциальную задачу \eqref{eqn:source} во внутренних узлах с помощью отношения конечных разностей по следующей схеме (вводится сеточная функция $y_{i, j}, i = \overline{0, N_x}, j = \overline{0, N_z}$):

\begin{equation}
	\label{eqn:sle}
	\begin{gathered}
		\frac{y_{i+1, j} - 2y_{i,j} + y_{i-1, j}}{h_x^2} + \frac{y_{i, j+1} - 2y_{i,j} + y_{i, j-1}}{h_z^2} + \frac{f(x_i, z_j)}{k} + O(h_x^2 + h_z^2) = 0, \\
		i=\overline{1, N_x - 1}, \quad j = \overline{1, N_z - 1},
	\end{gathered}
\end{equation}

\noindent которая на шаблоне (рис. \ref{img:template}) имеет второй порядок по переменным $x$ и $z$, поскольку шаблон центрально симметричен.

\begin{figure}[ht]
	\centering
	\begin{tikzpicture}
		\begin{axis}[
			xlabel={$x$},
			ylabel={$z$},
			axis lines=middle,
			xmin=-1,
			xmax=+1,
			ymin=-1,
			ymax=+1,
			xtick={-1, 0.15, 1},
			xticklabels={$i-1; j$, $i; j$, $i+1; j$},
			ytick={-1, 1},
			yticklabels={$i; j-1$, $i; j+1$},
			tick style={draw=none},
			axis line style={opacity=0},
			]
			\addplot [mark=*] coordinates {(-1, 0) (1, 0)};
			\addplot [mark=*] coordinates {(0, -1) (0, 0)};
			\addplot [mark=*] coordinates {(0, 0) (0, 1)};
		\end{axis}
	\end{tikzpicture}
	\caption{Центрально-симметричный шаблон для уравнения Пуассона}
	\label{img:template}
\end{figure}

СЛАУ \eqref{eqn:sle} имеет пяти-диагональный вид (каждое уравнение содержит пять неизвестных и при соответствующей нумерации переменных матрица имеет ленточную структуру).
Решать её можно различными методами линейной алгебры, например, итерационными методами, методом матричной прогонки и~т.~п.

Рассмотрим разностно-итерационный метод Либмана (метод Гаусса — Зейделя) численного решения нашей задачи \cite{2}.
Для простоты изложения этого метода примем $h_x = h_z = h$, тогда из схемы \eqref{eqn:sle} получим

\begin{equation}
	\label{eqn:liebmann}
	\begin{gathered}
		\hat y_{i, j} = \frac14 \left(y_{i+1, j} + y_{i-1,j} + y_{i, j+1} + y_{i, j-1} + \frac{f(x_i, z_j)}{k}h^2 \right), \\
		i = \overline{1, N_x-1}, \quad j = \overline{1, N_z - 1},
	\end{gathered}
\end{equation}

\noindent где $\hat y_{i, j} \equiv y_{i, j}^{k + 1}$, $y_{i, j} \equiv y_{i, j}^{k}$, $k$ — номер итерации.

Линейно проинтерполируем граничные значения $u$ для начальной итерации.
Тогда $y_{i, j}^0 \equiv u_0$.

\begin{equation}
	\begin{gathered}
		y_{i, j}^1 = u_0 + \frac14 \left(\frac{f(x_i, z_j)}{k}h^2 \right), \\
		i = \overline{1, N_x-1}, \quad j = \overline{1, N_z - 1}.
	\end{gathered}
\end{equation}

Это распределение $y_{i, j}^1$ снова подставляется в \eqref{eqn:liebmann}, получая распределение $y_{i, j}^2$ и~т.~д.
Процесс Либмана прекращается, когда

\begin{equation}
	\abs{y^{k+1} - y^k} \leqslant \varepsilon, \quad \abs{y^k} = \max_{i, j} \abs{y_{i, j}^k},
\end{equation}

\noindent где $\varepsilon$ — наперёд заданная точность.

\subsection{Продольно-поперечная схема}

Добавим в эллиптическое уравнение координату по времени, чтобы получилось уравнение параболического типа с двумя пространственными переменными:

\begin{equation}
	\frac{\partial u}{\partial t} = \frac{\partial^2 u}{\partial x^2} + \frac{\partial^2 u}{\partial z^2} + \frac{f(x, z)}{k}.
\end{equation}

Для составления схемы, которая носит название продольно-поперченой, введём полуцелый слой $\overline{t} = t_m + \frac\tau2$.
Схема имеет вид

\begin{equation}
	\label{eqn:schema1}
	\frac{\overline y_{ij} - y_{ij}}{0,5\tau} = A_1\overline y_{ij} + A_2y_{ij} + \frac{f_{ij}}k,
\end{equation}

\begin{equation}
	\label{eqn:schema2}
	\frac{\hat y_{ij} - \overline y_{ij}}{0,5\tau} = A_1\overline y_{ij} + A_2\hat y_{ij} + \frac{f_{ij}}k,
\end{equation}

\noindent причём разностные операторы $A_1$, $A_2$ действуют каждый по своему направлению (по своей координате) и определяются выражениями

\begin{equation}
	A_1 y_{ij} = \frac{1}{h_x^2} (y_{i - 1, j} - 2y_{ij} + y_{i + 1, j}),
\end{equation}

\begin{equation}
	A_2 y_{ij} = \frac{1}{h_z^2} (y_{i, j - 1} - 2y_{ij} + y_{i, j + 1}).
\end{equation}

Здесь $1 \leqslant i \leqslant N_x - 1$, $1 \leqslant j \leqslant N_z - 1$.

Схема \eqref{eqn:schema1}, \eqref{eqn:schema2} реализуется следующим образом.
Вначале вычисляют решение на полуцелом слое согласно \eqref{eqn:schema1}.
В системе линейных уравнений \eqref{eqn:schema1} с трёхдиагональной матрицей неизвестными являются величины $\overline y_{ij}$, которые находят прогонкой по индексу $i$ (по координате $x$) для каждого фиксированного значения индекса $j$.
При найденном решении $\overline y_{ij}$ система \eqref{eqn:schema2} также является линейной системой уравнений с трёхдиагональной матрицей, в которой неизвестными выступают $\hat y_{ij}$.
Решение $\hat y_{ij}$ находят прогонкой по индексу $j$ (по координате $z$) для каждого индекса $i$ \cite{1}.

Относительно аппроксимации и устойчивости продольно-поперечной схемы следует отметить, что схема \eqref{eqn:schema1}, \eqref{eqn:schema2} равномерно и безусловно устойчива по начальным данным и по правой части и аппроксимирует задачу на равномерных сетках с погрешностью $O(\tau^2 + h_x^2 + h_z^2)$.

Приведём уравнение \eqref{eqn:schema1} к стандартному виду для прогонки

\begin{equation}
	\frac{2(\overline y_{ij} - y_{ij})}{\tau} = \frac{1}{h_x^2} (\overline y_{i - 1, j} - 2\overline y_{ij} + \overline y_{i + 1, j}) + \frac{1}{h_z^2} (y_{i, j - 1} - 2y_{ij} + y_{i, j + 1}) + \frac{f_{ij}}k,
\end{equation}

\begin{equation}
	\frac{1}{h_x^2}\overline y_{i - 1, j} - 2 \left(\frac1{h_x^2} + \frac1\tau \right)\overline y_{ij} + \frac{1}{h_x^2}\overline y_{i + 1, j}
	= -\frac{1}{h_z^2}y_{i, j - 1} + 2\left(\frac1{h_z^2} - \frac1\tau\right)y_{ij} - \frac{1}{h_z^2}y_{i, j + 1} - \frac{f_{ij}}k,
\end{equation}

\begin{equation}
	\tau h_z^2 \overline y_{i - 1, j} - 2h_z^2(\tau + h_x^2)\overline y_{ij} + \tau h_z^2 \overline y_{i + 1, j}
	= h_x^2 \left( 2(\tau - h_z^2)y_{ij} - \tau \left( y_{i, j - 1} + y_{i, j + 1} + \frac{h_z^2 f_{ij}}k \right) \right)
\end{equation}

Коэффициенты для метода прогонки:

\begin{equation}
	\begin{dcases}
		A_i = \tau h_z^2, \\
		B_i = -2h_z^2(\tau + h_x^2), \\
		C_i = \tau h_z^2, \\
		D_i = h_x^2 \left( 2(\tau - h_z^2)y_{ij} - \tau \left( y_{i, j - 1} + y_{i, j + 1} + \frac{h_z^2 f_{ij}}k \right) \right), \\
		i = \overline{1, N_x - 1}. \\
	\end{dcases}
\end{equation}

Краевые условия:

\begin{equation}
	\begin{aligned}
		A_0     &= 0, & B_0     &= 1, & C_0     &= 0, & D_0     &= u_0, \\
		A_{N_x} &= 0, & B_{N_x} &= 1, & C_{N_x} &= 0, & D_{N_x} &= u_0. \\
	\end{aligned}
\end{equation}


Приведём уравнение \eqref{eqn:schema2} к стандартному виду для прогонки

\begin{equation}
	\frac{2(\hat y_{ij} - \overline y_{ij})}{\tau} = \frac{1}{h_x^2} (\overline y_{i - 1, j} - 2\overline y_{ij} + \overline y_{i + 1, j}) + \frac{1}{h_z^2} (\hat y_{i, j - 1} - 2\hat y_{ij} + \hat y_{i, j + 1}) + \frac{f_{ij}}k,
\end{equation}

\begin{equation}
	\frac1{h_z^2} \hat y_{i, j - 1} - 2 \left(\frac1{h_z^2} + \frac1\tau \right)\hat y_{ij} + \frac1{h_z^2}\hat y_{i, j + 1}
	= -\frac1{h_x^2}\overline y_{i - 1, j} + 2\left(\frac1{h_x^2} - \frac1\tau\right)\overline y_{ij} - \frac1{h_x^2}y_{i + 1, j} - \frac{f_{ij}}k,
\end{equation}

\begin{equation}
	\tau h_x^2 \hat y_{i, j - 1} - 2h_x^2(\tau + h_z^2)\hat y_{ij} + \tau h_x^2 \hat y_{i, j + 1}
	= h_z^2 \left( 2(\tau - h_x^2)\overline y_{ij} - \tau \left( \overline y_{i - 1, j} + \overline y_{i + 1, j} + \frac{h_x^2 f_{ij}}k \right) \right)
\end{equation}

Коэффициенты для метода прогонки:

\begin{equation}
	\begin{dcases}
		A_j = \tau h_x^2, \\
		B_j = -2h_x^2(\tau + h_z^2), \\
		C_j = \tau h_x^2, \\
		D_j = h_z^2 \left( 2(\tau - h_x^2)\overline y_{ij} - \tau \left( \overline y_{i - 1, j} + \overline y_{i + 1, j} + \frac{h_x^2 f_{ij}}k \right) \right), \\
		j = \overline{1, N_z - 1}. \\
	\end{dcases}
\end{equation}

Краевые условия:

\begin{equation}
	\begin{aligned}
		A_0     &= 0, & B_0     &= 1, & C_0     &= 0, & D_0     &= u_0, \\
		A_{N_z} &= 0, & B_{N_z} &= 1, & C_{N_z} &= 0, & D_{N_z} &= u_0. \\
	\end{aligned}
\end{equation}

\section{Вероятностный метод}

Теперь применим статистический метод решения уравнения Пуассона \eqref{eqn:source}.

Покроем область $0 \leqslant x \leqslant a, 0 \leqslant z \leqslant b$ квадратной сеткой с шагом $h$.

Из каждого узла сетки будем моделировать случайное блуждание частиц.
Находясь во внутреннем узле $y_{ij}$, частица $M$ может с равной вероятностью уйти либо влево, либо вправо, либо вверх, либо вниз.
Блуждание частицы $M$ заканчивается, как только она выходит на границу области.

Решение уравнения \eqref{eqn:source} в результате моделирования стохастического блуждания частиц имеет вид

\begin{equation}
	y(x, z) = \frac{1}{N_p} \sum_{j=1}^{N_p} y_b(j) + \left( \frac{h^2}{4k} \right) \frac{1}{N_p} \sum_{j=1}^{N_p} \overline f_jM_j,
\end{equation}

\noindent где

\begin{itemize}
	\item $N_p$ — количество разыгрываемых частиц из узла;
	\item $y_b(j)$ — граничное значение функции $y$ для $j$-й частицы;
	\item $\overline f_j$ — среднее значение функции $f$ по всем узлам траектории $j$-й частицы;
	\item $M_j$ — количество узлов в траектории $j$-й частицы.
\end{itemize}

\section{Сравнительный анализ методов решения}

Для определённости положим $f_0 = 100$, $\beta = -0.001$, $k = 2,36$.

Моделирование производилось на компьютере MacBook Pro 2019 со следующими характеристиками:

\begin{itemize}
	\item ОС: macOS Catalina 10.15.7,
	\item процессор: 1,4 ГГц 4-ядерный Intel Core i5,
	\item память: 8 ГБ 2133 МГц LPDDR3.
\end{itemize}

Сравним результаты моделирования на сетке с шагом $h_x = h_z = h = 1$ см, $\tau = 1$ с, $\varepsilon = 0,0001$, $N_p = 5000$.
В таблицах \ref{tbl:adi}—\ref{tbl:probability} представлены температурные поля, полученные продольно-поперечным методом, методом Либмана и вероятностным методом соответственно.
В таблице \ref{tbl:diff12} представлена разница значений в таблицах \ref{tbl:adi} и \ref{tbl:liebman}, в таблице \ref{tbl:diff13} — разница значений в таблицах \ref{tbl:adi} и \ref{tbl:probability}

\begin{table}[ht]
	\small
	\caption{Температурное поле, полученное продольно-поперечным методом}
	\label{tbl:adi}
	\begin{tabular}{|l|l|l|l|l|l|l|l|l|l|l|}
		\hline
		300 & 300     & 300     & 300     & 300     & 300     & 300     & 300     & 300     & 300     & 300 \\ \hline
		300 & 349.322 & 382.246 & 403.667 & 415.839 & 419.795 & 415.839 & 403.667 & 382.246 & 349.322 & 300 \\ \hline
		300 & 382.246 & 439.312 & 476.848 & 498.206 & 505.144 & 498.206 & 476.848 & 439.312 & 382.246 & 300 \\ \hline
		300 & 403.667 & 476.848 & 525.356 & 553.028 & 562.022 & 553.028 & 525.356 & 476.848 & 403.667 & 300 \\ \hline
		300 & 415.839 & 498.206 & 553.028 & 584.362 & 594.552 & 584.362 & 553.028 & 498.206 & 415.839 & 300 \\ \hline
		300 & 419.795 & 505.144 & 562.022 & 594.552 & 605.133 & 594.552 & 562.022 & 505.144 & 419.795 & 300 \\ \hline
		300 & 415.839 & 498.206 & 553.028 & 584.362 & 594.552 & 584.362 & 553.028 & 498.206 & 415.839 & 300 \\ \hline
		300 & 403.667 & 476.848 & 525.356 & 553.028 & 562.022 & 553.028 & 525.356 & 476.848 & 403.667 & 300 \\ \hline
		300 & 382.246 & 439.312 & 476.848 & 498.206 & 505.144 & 498.206 & 476.848 & 439.312 & 382.246 & 300 \\ \hline
		300 & 349.322 & 382.246 & 403.667 & 415.839 & 419.795 & 415.839 & 403.667 & 382.246 & 349.322 & 300 \\ \hline
		300 & 300     & 300     & 300     & 300     & 300     & 300     & 300     & 300     & 300     & 300 \\ \hline
	\end{tabular}
\end{table}

\begin{table}[ht]
	\small
	\caption{Температурное поле, полученное методом Либмана}
	\label{tbl:liebman}
	\begin{tabular}{|l|l|l|l|l|l|l|l|l|l|l|}
		\hline
		300 & 300     & 300     & 300     & 300     & 300     & 300     & 300     & 300     & 300     & 300 \\ \hline
		300 & 349.345 & 382.289 & 403.726 & 415.908 & 419.868 & 415.908 & 403.726 & 382.289 & 349.345 & 300 \\ \hline
		300 & 382.289 & 439.394 & 476.961 & 498.339 & 505.283 & 498.339 & 476.961 & 439.394 & 382.289 & 300 \\ \hline
		300 & 403.726 & 476.961 & 525.511 & 553.211 & 562.214 & 553.211 & 525.511 & 476.961 & 403.726 & 300 \\ \hline
		300 & 415.908 & 498.339 & 553.211 & 584.577 & 594.778 & 584.577 & 553.211 & 498.339 & 415.908 & 300 \\ \hline
		300 & 419.868 & 505.283 & 562.214 & 594.778 & 605.371 & 594.778 & 562.214 & 505.283 & 419.868 & 300 \\ \hline
		300 & 415.908 & 498.339 & 553.211 & 584.577 & 594.778 & 584.577 & 553.211 & 498.339 & 415.908 & 300 \\ \hline
		300 & 403.726 & 476.961 & 525.511 & 553.211 & 562.214 & 553.211 & 525.511 & 476.961 & 403.726 & 300 \\ \hline
		300 & 382.289 & 439.394 & 476.961 & 498.339 & 505.283 & 498.339 & 476.961 & 439.394 & 382.289 & 300 \\ \hline
		300 & 349.345 & 382.289 & 403.726 & 415.908 & 419.868 & 415.908 & 403.726 & 382.289 & 349.345 & 300 \\ \hline
		300 & 300     & 300     & 300     & 300     & 300     & 300     & 300     & 300     & 300     & 300 \\ \hline
	\end{tabular}
\end{table}

\begin{table}[ht]
	\small
	\caption{Температурное поле, полученное вероятностным методом}
	\label{tbl:probability}
	\begin{tabular}{|l|l|l|l|l|l|l|l|l|l|l|}
		\hline
		300 & 300     & 300     & 300     & 300     & 300     & 300     & 300     & 300     & 300     & 300 \\ \hline
		300 & 349.803 & 382.492 & 404.371 & 419.301 & 418.625 & 415.562 & 400.087 & 382.571 & 351.737 & 300 \\ \hline
		300 & 384.997 & 442.545 & 473.549 & 497.889 & 507.576 & 499.431 & 474.139 & 439.467 & 382.811 & 300 \\ \hline
		300 & 405.146 & 478.473 & 524.138 & 551.763 & 564.484 & 552.102 & 526.34  & 476.475 & 401.646 & 300 \\ \hline
		300 & 414.462 & 498.766 & 552.034 & 584.046 & 594.611 & 583.298 & 551.826 & 497.87  & 417.096 & 300 \\ \hline
		300 & 415.855 & 504.978 & 566.014 & 596.944 & 606.235 & 595.045 & 562.636 & 505.849 & 417.98  & 300 \\ \hline
		300 & 416.745 & 502.088 & 554.535 & 585.253 & 596.061 & 585.178 & 552.745 & 495.924 & 414.075 & 300 \\ \hline
		300 & 401.095 & 476.578 & 526.036 & 552.892 & 559.744 & 556.223 & 524.868 & 479.992 & 405.338 & 300 \\ \hline
		300 & 386.053 & 440.889 & 477.076 & 496.92  & 502.109 & 500.311 & 478.829 & 437.964 & 382.543 & 300 \\ \hline
		300 & 349.484 & 382.176 & 405.716 & 415.766 & 414.571 & 415.185 & 406.072 & 383.099 & 352.097 & 300 \\ \hline
		300 & 300     & 300     & 300     & 300     & 300     & 300     & 300     & 300     & 300     & 300 \\ \hline
	\end{tabular}
\end{table}

\begin{table}[ht]
	\small
	\caption{Поэлементная разница таблиц \ref{tbl:adi} и \ref{tbl:liebman}}
	\label{tbl:diff12}
	\begin{tabular}{|l|l|l|l|l|l|l|l|l|l|l|}
		\hline
		0 & 0       & 0       & 0       & 0       & 0       & 0       & 0       & 0       & 0       & 0 \\ \hline
		0 & -0.0227 & -0.0433 & -0.0594 & -0.0698 & -0.0734 & -0.0698 & -0.0593 & -0.0431 & -0.0227 & 0 \\ \hline
		0 & -0.0432 & -0.0821 & -0.1130 & -0.1328 & -0.1397 & -0.1328 & -0.1130 & -0.0821 & -0.0432 & 0 \\ \hline
		0 & -0.0594 & -0.1130 & -0.1554 & -0.1827 & -0.1920 & -0.1827 & -0.1553 & -0.1130 & -0.0594 & 0 \\ \hline
		0 & -0.0698 & -0.1328 & -0.1827 & -0.2147 & -0.2258 & -0.2147 & -0.1827 & -0.1328 & -0.0698 & 0 \\ \hline
		0 & -0.0734 & -0.1396 & -0.1921 & -0.2257 & -0.2372 & -0.2257 & -0.1921 & -0.1396 & -0.0734 & 0 \\ \hline
		0 & -0.0698 & -0.1328 & -0.1827 & -0.2147 & -0.2258 & -0.2147 & -0.1827 & -0.1328 & -0.0698 & 0 \\ \hline
		0 & -0.0594 & -0.1130 & -0.1554 & -0.1827 & -0.1920 & -0.1827 & -0.1554 & -0.1130 & -0.0594 & 0 \\ \hline
		0 & -0.0432 & -0.0821 & -0.1130 & -0.1328 & -0.1397 & -0.1328 & -0.1130 & -0.0821 & -0.0432 & 0 \\ \hline
		0 & -0.0227 & -0.0433 & -0.0594 & -0.0698 & -0.0734 & -0.0698 & -0.0593 & -0.0431 & -0.0227 & 0 \\ \hline
		0 & 0       & 0       & 0       & 0       & 0       & 0       & 0       & 0       & 0       & 0 \\ \hline
	\end{tabular}
\end{table}

\begin{table}[H]
	\small
	\caption{Поэлементная разница таблиц \ref{tbl:adi} и \ref{tbl:probability}}
	\label{tbl:diff13}
	\begin{tabular}{|l|l|l|l|l|l|l|l|l|l|l|}
		\hline
		0 & 0       & 0       & 0       & 0       & 0       & 0       & 0       & 0       & 0       & 0 \\ \hline
		0 & -0.4807 & -0.2463 & -0.7046 & -3.4620 & 1.1701  & 0.2765  & 3.5800  & -0.3253 & -2.4147 & 0 \\ \hline
		0 & -2.7512 & -3.2329 & 3.2989  & 0.3178  & -2.4320 & -1.2246 & 2.7091  & -0.1548 & -0.5650 & 0 \\ \hline
		0 & -1.4789 & -1.6246 & 1.2173  & 1.2659  & -2.4618 & 0.9264  & -0.9844 & 0.3736  & 2.0206  & 0 \\ \hline
		0 & 1.3766  & -0.5593 & 0.9949  & 0.3165  & -0.0591 & 1.0640  & 1.2022  & 0.3362  & -1.2569 & 0 \\ \hline
		0 & 3.9394  & 0.1655  & -3.9923 & -2.3919 & -1.1016 & -0.4929 & -0.6146 & -0.7057 & 1.81451 & 0 \\ \hline
		0 & -0.9067 & -3.8815 & -1.5061 & -0.8906 & -1.5090 & -0.8161 & 0.2839  & 2.2827  & 1.76364 & 0 \\ \hline
		0 & 2.5712  & 0.2703  & -0.6805 & 0.1367  & 2.2779  & -3.1948 & 0.4872  & -3.1434 & -1.6711 & 0 \\ \hline
		0 & -3.8075 & -1.5768 & -0.2278 & 1.2865  & 3.0342  & -2.1043 & -1.9807 & 1.3480  & -0.2969 & 0 \\ \hline
		0 & -0.1614 & 0.0694  & -2.0491 & 0.0722  & 5.2242  & 0.6533  & -2.4052 & -0.8537 & -2.7752 & 0 \\ \hline
		0 & 0       & 0       & 0       & 0       & 0       & 0       & 0       & 0       & 0       & 0 \\ \hline
	\end{tabular}
\end{table}

Наибольшая разница в таблице \ref{tbl:diff12} по модулю не превосходит $0,2372$, в таблице \ref{tbl:diff13} — $3,9923$.
Таким образом, можно заключить, что программная реализация методов даёт приблизительно одинаковый результат.

В таблице \ref{tbl:time} представлено время работы методов в зависимости от некоторых значений шага сетки $h$ (и шага по времени $\tau$ для продольно-поперечного метода).
$t_{final}$ в таблице обозначает модельное время до установления стационарного режима.
Количество итераций в продольно-поперечном методе равно удвоенному (если считать расчёт промежуточного слоя $\overline t$ за итерацию) отношению $t_{final}$ к $\tau$.

\begin{table}[H]
	\small
	\caption{Время работы методов}
	\label{tbl:time}
	\begin{tabular}{|l|l|ll|ll|l|}
		\hline
		\multicolumn{1}{|c|}{}                        & \multicolumn{1}{c|}{}                      & \multicolumn{2}{c|}{\textbf{Продольно-поперечный}}                           & \multicolumn{2}{c|}{\textbf{Либмана}}                                                                            & \multicolumn{1}{c|}{\textbf{Вероятностный}} \\ \cline{3-7}
		\multicolumn{1}{|c|}{\multirow{-2}{*}{$h$, см}}     & \multicolumn{1}{c|}{\multirow{-2}{*}{$\tau$, с}} & \multicolumn{1}{c|}{$t_{final}$, с}                     & \multicolumn{1}{c|}{Время, мс} & \multicolumn{1}{c|}{Итераций}                                        & \multicolumn{1}{c|}{Время, мс}                  & \multicolumn{1}{c|}{Время, мс}                    \\ \hline
		\rowcolor[HTML]{EFEFEF}
		1                                             & 1                                          & \multicolumn{1}{l|}{\cellcolor[HTML]{EFEFEF}37}   & 1.009                          & \multicolumn{1}{l|}{\cellcolor[HTML]{EFEFEF}233}                     & 0.225                                           & 194.35                                            \\ \hline
		\rowcolor[HTML]{ECF4FF}
		\cellcolor[HTML]{ECF4FF}                      & 1                                          & \multicolumn{1}{l|}{\cellcolor[HTML]{ECF4FF}37}   & 4.764                          & \multicolumn{1}{l|}{\cellcolor[HTML]{ECF4FF}}                        & \cellcolor[HTML]{ECF4FF}                        & \cellcolor[HTML]{ECF4FF}                          \\ \cline{2-4}
		\rowcolor[HTML]{ECF4FF}
		\multirow{-2}{*}{\cellcolor[HTML]{ECF4FF}0.5} & 0.5                                        & \multicolumn{1}{l|}{\cellcolor[HTML]{ECF4FF}32.5} & 3.57668                        & \multicolumn{1}{l|}{\multirow{-2}{*}{\cellcolor[HTML]{ECF4FF}820}}   & \multirow{-2}{*}{\cellcolor[HTML]{ECF4FF}4.432} & \multirow{-2}{*}{\cellcolor[HTML]{ECF4FF}923.74}  \\ \hline
		\rowcolor[HTML]{EFEFEF}
		0.2                                           & 0.2                                        & \multicolumn{1}{l|}{\cellcolor[HTML]{EFEFEF}27.6} & 26.0888                        & \multicolumn{1}{l|}{\cellcolor[HTML]{EFEFEF}4249}                    & 113.179                                         & 7007                                              \\ \hline
		\rowcolor[HTML]{ECF4FF}
		\cellcolor[HTML]{ECF4FF}                      & 1                                          & \multicolumn{1}{l|}{\cellcolor[HTML]{ECF4FF}37}   & 22.700                         & \multicolumn{1}{l|}{\cellcolor[HTML]{ECF4FF}}                        & \cellcolor[HTML]{ECF4FF}                        & \cellcolor[HTML]{ECF4FF}                          \\ \cline{2-4}
		\rowcolor[HTML]{ECF4FF}
		\multirow{-2}{*}{\cellcolor[HTML]{ECF4FF}0.1} & 0.1                                        & \multicolumn{1}{l|}{\cellcolor[HTML]{ECF4FF}24}   & 160.63                         & \multicolumn{1}{l|}{\multirow{-2}{*}{\cellcolor[HTML]{ECF4FF}13122}} & \multirow{-2}{*}{\cellcolor[HTML]{ECF4FF}1458}  & \multirow{-2}{*}{\cellcolor[HTML]{ECF4FF}27305}   \\ \hline
	\end{tabular}
\end{table}

Таким образом, из таблицы \ref{tbl:time} видно, что алгоритмы, основанные на конечно-разностной схеме решения дифференциального уравнения \eqref{eqn:source}, являются более быстрыми по сравнению с вероятностным методом.
Также можно заметить, что продольно-поперечная схема решается методом матричной прогонки явно быстрее, чем центрально-симметричная схема разностно-итерационным методом Либмана, несмотря на то, что для малых $N < 20$ ($h > 0,5$) метод Либмана выигрывает по скорости.

\section*{Заключение}

В ходе численного решения стационарного многомерного эллиптического уравнения теплопроводности выведены математические соотношения двух конечно-разностных алгоритмов: метода Либмана для центрально-симметричной схемы и метода прогонки для продольно-поперечной схемы; а также вероятностного метода. Проведён анализ программных реализаций методов на быстродействие. Выяснено, что наибольший интерес в плане производительности представляет матричная прогонка продольно-поперечная схемы, однако для сеток с малым количеством узлов (большим шагом), подойдёт и метод Либмана. Вероятностный метод является интересным с точки зрения получения численного решения, однако для серьёзных математических расчётов не подлежит.

\begin{thebibliography}{5}
	\bibitem{1} Градов В. М. Курс лекций по моделированию. МГТУ им. Н. Э. Баумана.
	\bibitem{2} Абдурагимов Э. И. Метод сеток решения задачи Дирихле для уравнения Пуассона. ДГУ
\end{thebibliography}

\noindent \textbf{Керимов Ахмед Шахович} — студент, МГТУ им. Н. Э. Баумана, кафедра «Программное обеспечение ЭВМ и информационные технологии».


\end{document}
