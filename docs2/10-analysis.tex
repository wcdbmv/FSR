%% Методические указания к выполнению, оформлению и защите выпускной квалификационной работы бакалавра
%% 2.4 Аналитический раздел
%%
%% В данном разделе расчётно-пояснительной записки проводится анализ предметной области и выделяется основной объект исследования.
%% Если формализовать предметную область с помощью математической модели не удаётся и при этом она сложна для понимания, то для отображения происходящих в ней процессов необходимо использовать методологию IDEF0, а для описания сущностей предметной области и взаимосвязей между ними — ER-модель.
%%
%% Затем выполняется обзор существующих методов и алгоритмов решения идентифицированной проблемы предметной области (опять же с обязательными ссылками на научные источники: монографии, статьи и др.) и их программных реализаций (при наличии), анализируются достоинства и недостатки каждого из них.
%% Выполненный обзор должен позволить объективно оценить актуальное состояние изучаемой проблемы.
%% Результаты проведённого анализа по возможности классифицируются и оформляются в табличной форме.
%%
%% На основе выполненного анализа обосновывается необходимость разработки нового или адаптации существующего метода или алгоритма.
%%
%% Если же целью анализа являлся отбор (на основе чётко сформулированных критериев) тех методов и алгоритмов, которые наиболее эффективно решают поставленную задачу, то форма представления результата должна подтвердить обоснованность сделанного выбора, в том числе — полноту и корректность предложенных автором критериев отбора.
%%
%% Одним из основных выводов аналитического раздела должно стать формализованное описание проблемы предметной области, на решение которой будет направлен данный проект, включающее в себя:
%% — описание входных и выходных данных;
%% — указание ограничений, в рамках которых будет разработан новый, адаптирован существующий или просто реализован метод или алгоритм;
%% — описание критериев сравнения нескольких реализаций метода или алгоритма;
%% — описание способов тестирования разработанного, адаптированного или реализованного метода или алгоритма;
%% — описание функциональных требований к разрабатываемому программному обеспечению,
%% при этом в зависимости от направления работы отдельные пункты могут отсутствовать.
%%
%% Если в результате работы будет создано программное обеспечение, реализующее большое количество типичных способов взаимодействия с пользователем, необходимо каждый из этих способов описать с помощью диаграммы прецедентов [4, 5].
%%
%% Рекомендуемый объём аналитического раздела 25—30 страниц.

\section*{Постановка задачи}




\chapter{Аналитический раздел}

\section{Исходные данные}

Вариант 3. Математическая модель с постоянными коэффициентами $k(x, z) \equiv k$:

\begin{equation}
	\frac{\partial^2 u}{\partial x^2} + \frac{\partial^2 u}{\partial z^2} + \frac{f(x, z)}{k} = 0.
\end{equation}

На границах прямоугольной области $(0 < x < a, 0 < z < b)$.

\begin{center}
	\begin{tikzpicture}
		\begin{axis}[
			xlabel={$x$},
			ylabel={$z$},
			every axis x label/.style={at={(current axis.right of origin)},anchor=west},
			every axis y label/.style={at={(current axis.north west)},above=2mm},
			axis lines=left,
			xmin=0,
			xmax=1.1,
			ymin=0,
			ymax=1.25,
			xtick={0, 1},
			xticklabels={$0$, $a$},
			ytick={1},
			yticklabels={$b$},
			tick style={draw=none},
		]
			\addplot[
				mark=none,
				very thick,
			] coordinates{
				(0, 1)
				(1, 1)
			};

			\addplot[
				mark=none,
				very thick,
			] coordinates {
				(1, 0)
				(1, 1)
			};
		\end{axis}
	\end{tikzpicture}
\end{center}

Краевые условия:

\begin{equation}
	\begin{dcases}
		x = 0, & u(0, z) = u_0, \\
		x = a, & u(a, z) = u_0, \\
		z = 0, & u(x, 0) = u_0, \\
		z = b, & u(x, b) = u_0. \\
	\end{dcases}
\end{equation}

Значения коэффициентов задачи (все размерности согласованы).

Геометрические размеры $a = b = 10$ см.

$u_0 = 300$ К.

В качестве примера функции источников можно предложить распределение вида $f(x, z) = f_0 e^{\beta(x - a/2)^2(z - b/2)^2}$, параметры $f_0, \beta$ варьируются исходя из условия, чтобы максимум функции не превышал $3000$~K.

\section{Физическое содержание задачи}

Сформулированная математическая модель описывает двумерное температурное поле $u(x, z)$ в тонкой прямоугольной пластине с размерами $a \times b$.
Температура по толщине пластины (третьей координате) принимается постоянной.
Функция $f(x, z)$ представляет внутренние объёмные источники тепловыделения, например, за счёт поглощения излучения в полупрозрачном материале пластины.
Краевые условия в варианте 1 соответствуют нагружению объекта тепловым потоком $F_0$ с одной стороны $x = 0$, постоянным вдоль координаты $z$, и отводу тепла с трёх других сторон при заданной температуре окружающей среды $u_0$.
Можно считать, что пластина охлаждается воздухом или водой, температура которых равна $u_0$ с коэффициентами теплоотдачи $\alpha_i$, в общем случае своими для каждой из четырёх сторон.
Функция $k(x, z)$ является коэффициентом теплопроводности материала стержня.

\section{Результаты работы}

\begin{enumerate}
	\item Выбрать разностный метод и разработать алгоритм и программу для численного исследования модели в одном из вариантов уравнения в краевых условиях.
	\item Разработать алгоритм и программу для численного исследования модели вероятностным методом для тех же вариантов уравнения и краевых условий.
	\item Оптимизировать разработанные алгоритмы, проведя серии вычислительных экспериментов.
	\item Дать сравнительный анализ методов и алгоритмов, указать условия и области их преимуществ друг перед другом.
	\item Подготовить научную статью по результатам исследований.
\end{enumerate}


\section{Выводы}
