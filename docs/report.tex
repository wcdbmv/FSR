\documentclass[12pt, a4paper]{article}

\usepackage[T2A]{fontenc}
\usepackage[utf8]{inputenc}
\usepackage[main=russian, english]{babel}

\usepackage{microtype}
\sloppy


%% ГОСТ 7.32-2017
%% 6.1 Общие требования
%%
%% 6.1.1 Изложение текста и оформление отчёта выполняют в соответствии с требованиями настоящего стандарта.
%% Страницы текста отчёта о НИР и включённые в отчёт иллюстрации и таблицы должны соответствовать формату А4 по ГОСТ 9327.
%% Допускается применение формата А3 при наличии большого количества таблиц и иллюстраций данного формата.
%%
%% Отчёт о НИР должен быть выполнен любым печатным способом на одной стороне листа белой бумаги формата А4 через полтора интервала.

\usepackage[onehalfspacing]{setspace}

%% Допускается при подготовке заключительного отчёта о НИР печатать через один интервал, если отчёт имеет значительный объём (500 и более страниц).
%% Цвет шрифта должен быть чёрным, размер шрифта — не менее 12 пт.
%% Рекомендуемый тип шрифта для основного текста отчёта — Times New Roman.

%\usepackage{paratype}
%\renewcommand{\rmdefault}{PTSerif-TLF}
%\renewcommand{\ttdefault}{PTMono-TLF}

%% Полужирный шрифт применяют только для заголовков разделов и подразделов, заголовков структурных элементов.
%% Использование курсива допускается для обозначения объектов (биология, геология, медицина, нанотехнологии, генная инженерия и др.) и написания терминов (например, in vivo, in vitro) и иных объектов и терминов на латыни.
%%
%% Для акцентирования внимания может применяться выделение текста с помощью шрифта иного начертания, чем шрифт основного текста, но того же кегля и гарнитуры.
%% Разрешается для написания определённых терминов, формул, теорем применять шрифты разной гарнитуры.
%%
%% Текст отчёта следует печатать, соблюдая следующие размеры полей: левое — 30 мм, правое — 15 мм, верхнее и нижнее — 20 мм.

\newcommand{\PageLeftMargin}{30mm}
\newcommand{\PageRightMargin}{15mm}
\newcommand{\PageTopMargin}{20mm}
\newcommand{\PageBottomMargin}{20mm}

\usepackage[
	left=\PageLeftMargin,
	right=\PageRightMargin,
	top=\PageTopMargin,
	bottom=\PageBottomMargin,
]{geometry}

%% Абзацный отступ должен быть одинаковым по всему тексту отчёта и равен 1,25 см.

\usepackage{indentfirst}
\setlength{\parindent}{1.25cm}

\setlength{\parskip}{1ex}

%%
%% 6.1.2 Вне зависимости от способа выполнения отчёта качество напечатанного текста и оформления иллюстраций, таблиц, распечаток программ должно удовлетворять требованию их чёткого воспроизведения.
%%
%% 6.1.3 При выполнении отчёта о НИР необходимо соблюдать равномерную плотность и чёткость изображения по всему отчёту.
%% Все линии, буквы, цифры и знаки должны иметь одинаковую контрастность по всему тексту отчёта.
%%
%% 6.1.4 Фамилии, наименования учреждений, организаций, фирм, наименования изделий и другие имена собственные в отчёте приводят на языке оригинала.
%% Допускается транслитерировать имена собственные и приводить наименования организаций в переводе на язык отчёта с добавлением (при первом упоминании) оригинального названия по ГОСТ 7.79.
%%
%% 6.1.5 Сокращения слов и словосочетаний на русском, белорусском [1] и иностранных европейских языках оформляют в соответствии с требованиями ГОСТ 7.11, ГОСТ 7.12.
%%   [1] Для Республики Беларусь применим СТБ 7.12.
%%
%% 6.2 Построение отчёта
%%
%% 6.2.1 Наименования структурных элементов отчёта: "СПИСОК ИСПОЛНИТЕЛЕЙ", "РЕФЕРАТ", "СОДЕРЖАНИЕ", "ТЕРМИНЫ И ОПРЕДЕЛЕНИЯ", "ПЕРЕЧЕНЬ СОКРАЩЕНИЙ И ОБОЗНАЧЕНИЙ", "ВВЕДЕНИЕ", "ЗАКЛЮЧЕНИЕ", "СПИСОК ИСПОЛЬЗОВАННЫХ ИСТОЧНИКОВ", "ПРИЛОЖЕНИЕ" служат заголовками структурных элементов отчёта.

\AtBeginDocument{\renewcommand\contentsname{Содержание}}

%%
%% Заголовки структурных элементов следует располагать в середине строки без точки в конце, прописными буквами, не подчёркивая.
%% Каждый структурный элемент и каждый раздел основной части отчёта начинают с новой страницы.

\usepackage{titlesec}

\newcommand{\ChapterLeftMargin}{}
\newcommand{\ChapterFormat}{}
\newcommand{\ChapterSep}{}
\newcommand{\ChapterBeforeCode}{}

\newcommand{\SetMainPartChapterSettings}{%
	\renewcommand{\ChapterLeftMargin}{\parindent}%
	\renewcommand{\ChapterFormat}{\bfseries}%
	\renewcommand{\ChapterSep}{1em}%
	\renewcommand{\ChapterBeforeCode}{}%
}

\newcommand{\SetStructuralElementChapterSettings}{%
	\renewcommand{\ChapterLeftMargin}{0pt}%
	\renewcommand{\ChapterFormat}{\filcenter\bfseries}%
	\renewcommand{\ChapterSep}{}%
	\renewcommand{\ChapterBeforeCode}{\MakeUppercase}%
}

\SetStructuralElementChapterSettings

\titlespacing*{\chapter}{\ChapterLeftMargin}{-30pt}{8pt}
\titleformat{\chapter}[block]{\ChapterFormat}{\thechapter}{\ChapterSep}{\ChapterBeforeCode}{}

\newcommand{\StructuralElement}[1]{%
	\chapter*{#1}%
	\addcontentsline{toc}{chapter}{\MakeUppercase{#1}}%
}

\newenvironment{MainPart}%
	{\SetMainPartChapterSettings}%
	{\SetStructuralElementChapterSettings}

%%
%% 6.2.2 Основную часть отчёта следует делить на разделы, подразделы и пункты.
%% Пункты при необходимости могут делиться на подпункты.
%% Разделы и подразделы отчёта должны иметь заголовки.
%% Пункты и подпункты, как правило, заголовков не имеют.
%%
%% 6.2.3 Заголовки разделов и подразделов основной части отчёта следует начинать с абзацного отступа и размещать после порядкового номера, печатать с прописной буквы, полужирным шрифтом, не подчёркивать, без точки в конце.
%% Пункты и подпункты могут иметь только порядковый номер без заголовка, начинающийся с абзацного отступа.

\titlespacing*{\section}{\parindent}{*4}{*2}
\titlespacing*{\subsection}{\parindent}{*4}{*4}
\titlespacing*{\subsubsection}{\parindent}{*4}{*4}
\titlespacing*{\paragraph}{\parindent}{*0}{*1}
\titleformat{\section}{\bfseries}{\thesection}{1em}{}{}{}
\titleformat{\subsection}{\bfseries}{\thesubsection}{1em}{}{}{}
\titleformat{\subsubsection}{\bfseries}{\thesubsubsection}{1em}{}{}{}

%%
%% 6.2.4 Если заголовок включает несколько предложений, их разделяют точками.
%% Переносы слов в заголовках не допускаются.
%%
%% 6.3 Нумерация страниц отчёта
%%
%% 6.3.1 Страницы отчёта следует нумеровать арабскими цифрами, соблюдая сквозную нумерацию по всему тексту отчёта, включая приложения.
%% Номер страницы проставляется в центре нижней части страницы без точки.
%% Приложения, которые приведены в отчёте о НИР и имеющие собственную нумерацию, допускается не перенумеровать.
%%
%% 6.3.2 Титульный лист включают в общую нумерацию страниц отчёта.
%% Номер страницы на титульном листе не проставляют.
%%
%% 6.3.3 Иллюстрации и таблицы, расположенные на отдельных листах, включают в общую нумерацию страниц отчёта.
%% Иллюстрации и таблицы на листе формата А3 учитывают как одну страницу.
%%
%% 6.4 Нумерация разделов, подразделов, пунктов, подпунктов и книг отчёта
%%
%% 6.4.1 Разделы должны иметь порядковые номера в пределах всего отчёта, обозначенные арабскими цифрами без точки и расположенные с абзацного отступа.
%% Подразделы должны иметь нумерацию в пределах каждого раздела.
%% Номер подраздела состоит из номеров раздела и подраздела, разделённых точкой.
%% В конце номера подраздела точка не ставится.
%% Разделы, как и подразделы, могут состоять из одного или нескольких пунктов.
%%
%% 6.4.2 Если отчёт не имеет подразделов, то нумерация пунктов в нем должна быть в пределах каждого раздела и номер пункта должен состоять из номеров раздела и пункта, разделённых точкой.
%% В конце номера пункта точка не ставится.
%%
%% Если отчёт имеет подразделы, то нумерация пунктов должна быть в пределах подраздела и номер пункта должен состоять из номеров раздела, подраздела и пункта, разделённых точками.
%%
%% 6.4.3 Если раздел или подраздел состоит из одного пункта, то пункт не нумеруется.
%%
%% 6.4.4 Если текст отчёта подразделяется только на пункты, они нумеруются порядковыми номерами в пределах отчёта.
%%
%% 6.4.5 Пункты при необходимости могут быть разбиты на подпункты, которые должны иметь порядковую нумерацию в пределах каждого пункта: 4.2.1.1, 4.2.1.2, 4.2.1.3 и т.д.

\setcounter{tocdepth}{3}
\setcounter{secnumdepth}{3}

%%
%% 6.4.6 Внутри пунктов или подпунктов могут быть приведены перечисления.
%% Перед каждым элементом перечисления следует ставить тире.

\renewcommand{\labelitemi}{—}
\renewcommand{\labelitemii}{—}

%% При необходимости ссылки в тексте отчёта на один из элементов перечисления вместо тире ставят строчные буквы русского алфавита со скобкой, начиная с буквы "а" (за исключением букв ё, з, й, о, ч, ъ, ы, ь).

\renewcommand{\labelenumi}{\asbuk{enumi})}
\renewcommand{\labelenumii}{\arabic{enumii})}
\usepackage{enumitem}

\makeatletter
	\AddEnumerateCounter{\asbuk}{\@asbuk}{ю)}
\makeatother
\setlist{nosep, leftmargin=\parindent}

%% Простые перечисления отделяются запятой, сложные — точкой с запятой.
%%
%% При наличии конкретного числа перечислений допускается перед каждым элементом перечисления ставить арабские цифры, после которых ставится скобка.
%%
%% Перечисления приводятся с абзацного отступа в столбик.
%%
%% 6.4.7 Заголовки должны чётко и кратко отражать содержание разделов, подразделов.
%% Если заголовок состоит из двух предложений, их разделяют точкой.
%%
%% 6.4.8 Если отчёт состоит из двух и более книг, каждая книга должна иметь свой порядковый номер.
%% Номер каждой книги следует проставлять арабскими цифрами на титульном листе под указанием вида отчёта: "Книга 2".
%%
%% 6.5 Иллюстрации

\usepackage{graphicx}
\usepackage{float}

\usepackage[tableposition=top, singlelinecheck=false]{caption}
\usepackage{subcaption}

\DeclareCaptionLabelFormat{gostfigure}{Рисунок #2}
\DeclareCaptionLabelFormat{gosttable}{Таблица #2}
\DeclareCaptionLabelSeparator{gost}{~—~}
\captionsetup{labelsep=gost}
\captionsetup*[figure]{labelformat=gostfigure}
\captionsetup*[table]{labelformat=gosttable}
\renewcommand{\thesubfigure}{\asbuk{subfigure}}

\usepackage{multirow}
\usepackage[table,xcdraw]{xcolor}

%%
%% 6.5.1 Иллюстрации (чертежи, графики, схемы, компьютерные распечатки, диаграммы, фотоснимки) следует располагать в отчёте непосредственно после текста отчёта, где они упоминаются впервые, или на следующей странице (по возможности ближе к соответствующим частям текста отчёта).
%% На все иллюстрации в отчёте должны быть даны ссылки.
%% При ссылке необходимо писать слово "рисунок" и его номер, например: "в соответствии с рисунком 2" и т.д.

\newcommand{\imght}[3]{%
	\begin{figure}[ht]%
		\center{\includegraphics[#1]{inc/img/#2}}%
		\captionsetup{justification=centering}%
		\caption{#3}%
		\label{img:#2}%
	\end{figure}%
}

\newcommand{\imgH}[3]{%
	\begin{figure}[H]%
		\center{\includegraphics[#1]{inc/img/#2}}%
		\captionsetup{justification=centering}%
		\caption{#3}%
		\label{img:#2}%
	\end{figure}%
}

\usepackage{pgfplots}

%%
%% 6.5.2 Чертежи, графики, диаграммы, схемы, помещаемые в отчёте, должны соответствовать требованиям стандартов Единой системы конструкторской документации (ЕСКД).
%%
%% 6.5.3 Количество иллюстраций должно быть достаточным для пояснения излагаемого текста отчёта.
%% Не рекомендуется в отчёте о НИР приводить объёмные рисунки.
%%
%% 6.5.4 Иллюстрации, за исключением иллюстраций, приведённых в приложениях, следует нумеровать арабскими цифрами сквозной нумерацией.
%% Если рисунок один, то он обозначается: Рисунок 1.
%%   Пример — Рисунок 1 — Схема прибора
%%
%% 6.5.5 Иллюстрации каждого приложения обозначают отдельной нумерацией арабскими цифрами с добавлением перед цифрой обозначения приложения: Рисунок А.3.
%%
%% 6.5.6 Допускается нумеровать иллюстрации в пределах раздела отчёта.
%% В этом случае номер иллюстрации состоит из номера раздела и порядкового номера иллюстрации, разделённых точкой: Рисунок 2.1.
%%
%% 6.5.7 Иллюстрации при необходимости могут иметь наименование и пояснительные данные (подрисуночный текст).
%% Слово "Рисунок", его номер и через тире наименование помещают после пояснительных данных и располагают в центре под рисунком без точки в конце.
%%   Пример — Рисунок 2 — Оформление таблицы
%%
%% 6.5.8 Если наименование рисунка состоит из нескольких строк, то его следует записывать через один межстрочный интервал.
%% Наименование рисунка приводят с прописной буквы без точки в конце.
%% Перенос слов в наименовании графического материала не допускается.
%%
%% 6.6 Таблицы

\renewcommand{\arraystretch}{1.3}

\newcommand{\TableHeader}[2]{%
	\parbox{#1}{%
		\vspace{.5\baselineskip}%
		\centering{\textbf{#2}}%
		\vspace{.5\baselineskip}%
	}%
}

%%
%% 6.6.1 Цифровой материал должен оформляться в виде таблиц.
%% Таблицы применяют для наглядности и удобства сравнения показателей.
%%
%% 6.6.2 Таблицу следует располагать непосредственно после текста, в котором она упоминается впервые, или на следующей странице.
%%
%% На все таблицы в отчёте должны быть ссылки.
%% При ссылке следует печатать слово "таблица" с указанием её номера.
%%
%% 6.6.3 Наименование таблицы, при ее* наличии, должно отражать её содержание, быть точным, кратким.
%% Наименование следует помещать над таблицей слева, без абзацного отступа в следующем формате: Таблица Номер таблицы — Наименование таблицы.
%% Наименование таблицы приводят с прописной буквы без точки в конце.
%%   * Вероятно ошибка оригинала. Следует читать "его". - Примечание изготовителя базы данных.
%%
%% Если наименование таблицы занимает две строки и более, то его следует записывать через один межстрочный интервал.
%%
%% Таблицу с большим количеством строк допускается переносить на другую страницу.
%% При переносе части таблицы на другую страницу слово "Таблица", её номер и наименование указывают один раз слева над первой частью таблицы, а над другими частями также слева пишут слова "Продолжение таблицы" и указывают номер таблицы.
%%
%% При делении таблицы на части допускается её головку или боковик заменять соответственно номерами граф и строк.
%% При этом нумеруют арабскими цифрами графы и (или) строки первой части таблицы.
%% Таблица оформляется в соответствии с рисунком 1.
%%
%% 6.6.4 Таблицы, за исключением таблиц приложений, следует нумеровать арабскими цифрами сквозной нумерацией.
%%
%% Таблицы каждого приложения обозначаются отдельной нумерацией арабскими цифрами с добавлением перед цифрой обозначения приложения.
%% Если в отчёте одна таблица, она должна быть обозначена "Таблица 1" или "Таблица А.1" (если она приведена в приложении А).
%%
%% Допускается нумеровать таблицы в пределах раздела при большом объёме отчёта.
%% В этом случае номер таблицы состоит из номера раздела и порядкового номера таблицы, разделённых точкой: Таблица 2.3.
%%
%% 6.6.5 Заголовки граф и строк таблицы следует печатать с прописной буквы, а подзаголовки граф — со строчной буквы, если они составляют одно предложение с заголовком, или с прописной буквы, если они имеют самостоятельное значение.
%% В конце заголовков и подзаголовков таблиц точки не ставятся.
%% Названия заголовков и подзаголовков таблиц указывают в единственном числе.
%%
%% 6.6.6 Таблицы слева, справа, сверху и снизу ограничивают линиями.
%% Разделять заголовки и подзаголовки боковика и граф диагональными линиями не допускается.
%% Заголовки граф выравнивают по центру, а заголовки строк — по левому краю.
%%
%% Горизонтальные и вертикальные линии, разграничивающие строки таблицы, допускается не проводить, если их отсутствие не затрудняет пользование таблицей.
%%
%% 6.6.7 Текст, повторяющийся в строках одной и той же графы и состоящий из одиночных слов, заменяют кавычками.
%% Ставить кавычки вместо повторяющихся цифр, буквенно-цифровых обозначений, знаков и символов не допускается.
%%
%% Если текст повторяется, то при первом повторении его заменяют словами "то же", а далее кавычками.
%%
%% В таблице допускается применять размер шрифта меньше, чем в тексте отчёта.
%%
%% 6.7 Примечания и сноски

\renewcommand*{\thefootnote}{\arabic{footnote})}
\renewcommand{\footnoterule}{%
	\kern -3pt%
	\hrule width 40mm height .4pt%
	\kern 2.6pt%
}

%%
%% 6.7.1 Примечания приводят в отчёте, если необходимы пояснения или справочные данные к содержанию текста, таблиц или графического материала.
%%
%% 6.7.2 Слово "Примечание" следует печатать с прописной буквы с абзацного отступа, не подчёркивая.
%%
%% 6.7.3 Примечания следует помещать непосредственно после текстового, графического материала или таблицы, к которым относятся эти примечания.
%% Если примечание одно, то после слова "Примечание" ставится тире и текст примечания печатают с прописной буквы.
%% Одно примечание не нумеруется. Несколько примечаний нумеруют по порядку арабскими цифрами без точки.
%%
%% 6.7.4 При необходимости дополнительного пояснения в отчёте допускается использовать примечание, оформленное в виде сноски.
%% Знак сноски ставят без пробела непосредственно после того слова, числа, символа, предложения, к которому даётся пояснение.
%% Знак сноски указывается надстрочно арабскими цифрами.
%% Допускается вместо цифр использовать знак звёздочка - *.
%%
%% Сноску располагают с абзацного отступа в конце страницы, на которой приведено поясняемое слово (словосочетание или данные).
%% Сноску отделяют от текста короткой сплошной тонкой горизонтальной линией с левой стороны страницы.
%%
%% 6.8 Формулы и уравнения

\usepackage{amsmath}
\usepackage{amssymb}
\usepackage{commath}
\usepackage{icomma}
\usepackage{mathtools}

\DeclareMathOperator{\sign}{sign}

%%
%% 6.8.1 Уравнения и формулы следует выделять из текста в отдельную строку.
%% Выше и ниже каждой формулы или уравнения должно быть оставлено не менее одной свободной строки.
%% Если уравнение не умещается в одну строку, оно должно быть перенесено после знака равенства (=) или после знаков плюс (+), минус (-), умножения (х), деления (:) или других математических знаков.
%% На новой строке знак повторяется.
%% При переносе формулы на знаке, символизирующем операцию умножения, применяют знак "X".
%%
%% 6.8.2 Пояснение значений символов и числовых коэффициентов следует приводить непосредственно под формулой в той же последовательности, в которой они представлены в формуле.
%% Значение каждого символа и числового коэффициента необходимо приводить с новой строки.
%% Первую строку пояснения начинают со слова "где" без двоеточия с абзаца.
%%
%% 6.8.3 Формулы в отчёте следует располагать посередине строки и обозначать порядковой нумерацией в пределах всего отчёта арабскими цифрами в круглых скобках в крайнем правом положении на строке.
%% Одну формулу обозначают (1).
%%
%% 6.8.4 Ссылки в отчёте на порядковые номера формул приводятся в скобках: в формуле (1).
%%
%% 6.8.5 Формулы, помещаемые в приложениях, нумеруются арабскими цифрами в пределах каждого приложения с добавлением перед каждой цифрой обозначения приложения: (В.1).
%%
%% Допускается нумерация формул в пределах раздела.
%% В этом случае номер формулы состоит из номера раздела и порядкового номера формулы, разделённых точкой: (3.1).
%%
%% 6.9 Ссылки

\usepackage[unicode]{hyperref}
\hypersetup{hidelinks}

\usepackage{csquotes}
\usepackage[%
	backend=biber,
	bibencoding=utf8,
	language=auto,
	style=gost-numeric,
	sorting=none,
]{biblatex}
\addbibresource{91-references.bib}

% Режим доступа вместо URL в biblatex
\DeclareFieldFormat{url}{Режим доступа\addcolon\space\url{#1}}
% Дата обращения без сокращений в biblatex
\DeclareFieldFormat{urldate}{(дата обращения\addcolon\space\urldate{#1})}

%%
%% 6.9.1 В отчёте о НИР рекомендуется приводить ссылки на использованные источники.
%% При нумерации ссылок на документы, использованные при составлении отчёта, приводится сплошная нумерация для всего текста отчёта в целом или для отдельных разделов.
%% Порядковый номер ссылки (отсылки) приводят арабскими цифрами в квадратных скобках в конце текста ссылки.
%% Порядковый номер библиографического описания источника в списке использованных источников соответствует номеру ссылки.
%%
%% 6.9.2 Ссылаться следует на документ в целом или на его разделы и приложения.
%%
%% 6.9.3 При ссылках на стандарты и технические условия указывают их обозначение, при этом допускается не указывать год их утверждения при условии полного описания стандарта и технических условий в списке использованных источников в соответствии с ГОСТ 7.1.


%% for title-page
\usepackage{wrapfig}
%%
\makeatletter
	\def\vhrulefill#1{\leavevmode\leaders\hrule\@height#1\hfill\kern\z@}
\makeatother


%% for listings
\usepackage{algorithm}
\usepackage{algpseudocode}
%%
\renewcommand{\listalgorithmname}{Список алгоритмов}
\floatname{algorithm}{Алгоритм}
\algrenewcommand\algorithmicwhile{\textbf{До тех пор, пока}}
\algrenewcommand\algorithmicdo{\textbf{выполнять}}
\algrenewcommand\algorithmicrepeat{\textbf{Повторять}}
\algrenewcommand\algorithmicuntil{\textbf{Пока выполняется}}
\algrenewcommand\algorithmicend{\textbf{Конец}}
\algrenewcommand\algorithmicif{\textbf{Если}}
\algrenewcommand\algorithmicelse{\textbf{иначе}}
\algrenewcommand\algorithmicthen{\textbf{тогда}}
\algrenewcommand\algorithmicfor{\textbf{Цикл}}
\algrenewcommand\algorithmicforall{\textbf{Для всех}}
\algrenewcommand\algorithmicfunction{\textbf{Функция}}
\algrenewcommand\algorithmicprocedure{\textbf{Процедура}}
\algrenewcommand\algorithmicloop{\textbf{Зациклить}}
\algrenewcommand\algorithmicrequire{\textbf{Условия:}}
\algrenewcommand\algorithmicensure{\textbf{Обеспечивающие условия:}}
\algrenewcommand\algorithmicreturn{\textbf{Возвратить}}
\algrenewtext{EndWhile}{\textbf{Конец цикла}}
\algrenewtext{EndLoop}{\textbf{Конец зацикливания}}
\algrenewtext{EndFor}{\textbf{Конец цикла}}
\algrenewtext{EndFunction}{\textbf{Конец функции}}
\algrenewtext{EndProcedure}{\textbf{Конец процедуры}}
\algrenewtext{EndIf}{\textbf{Конец условия}}
\algrenewtext{EndFor}{\textbf{Конец цикла}}
\algrenewtext{BeginAlgorithm}{\textbf{Начало алгоритма}}
\algrenewtext{EndAlgorithm}{\textbf{Конец алгоритма}}
\algrenewtext{BeginBlock}{\textbf{Начало блока. }}
\algrenewtext{EndBlock}{\textbf{Конец блока}}
\algrenewtext{ElsIf}{\textbf{иначе если }}
%%
\usepackage{listings}
\usepackage{listingsutf8}
\usepackage{xcolor}
\lstset{
	basicstyle=\ttfamily,
	breakatwhitespace=true,
	breaklines=true,
	commentstyle=\color{gray},
	frame=single,
	inputencoding=utf8/koi8-r,
	keywordstyle=\color{blue}\bfseries,
	showstringspaces=false,
	stringstyle=\color{red},
	tabsize=8,
	xleftmargin=3pt,
	xrightmargin=3pt,
}
%%
\newcommand{\code}[1]{\texttt{#1}}

%% reduce indent in contents
%% https://tex.stackexchange.com/questions/409569/change-indent-in-standard-table-of-content-not-tocloft
\usepackage{etoolbox}
\makeatletter
	% \patchcmd{<cmd>}{<search>}{<replace>}{<success>}{<failure>}
	\patchcmd{\l@section}{1.5em}{1em}{}{}
	\patchcmd{\l@subsection}{3.8em}{2em}{}{}
	\patchcmd{\l@subsubsection}{7.0em}{3em}{}{}
\makeatother


\begin{document}


\noindent УДК~004.94

\hfill

\noindent \textbf{Сравнительный анализ разностного и вероятностного методов исследования математической модели, построенной на дифференциальном уравнении в частных производных эллиптического типа}

\noindent А.~Ш.~Керимов$^{1}$\hfill kerimov.edu@yandex.ru

\noindent $^{1}$МГТУ им.~Н.~Э.~Баумана, Москва, Россия

\hfill

\noindent \textbf{Аннотация}

\noindent Статья посвящена численным методам приближённого решения стационарного многомерного эллиптического уравнения теплопроводности.
Описаны математические соотношения конечно-разностного и вероятностного методов. Проведён анализ программных реализаций методов на быстродействие. Предложены алгоритмические улучшения вероятностного метода.

\noindent \textbf{Ключевые слова}

\noindent \textit{математическая физика, задача двумерной теплопроводности, численные методы, уравнение Пуассона, продольно-поперечная схема, матричная прогонка, вероятностный метод, метод Монте-Карло}

\hfill

\section*{Введение}

Прикладные проблемы приводят к необходимости решения краевых задач для уравнений с частными производными. Разработка приближённых методов их решения базируется на построении и исследовании численных методов решения краевых задач для базовых (основных, модельных) уравнений математической физики. В качестве таковых при рассмотрении уравнений второго порядка выделяются эллиптические, параболические и гиперболические уравнения.

\section{Постановка задачи}

Задана математическая модель с постоянными коэффициентами $k(x, z) \equiv k$ (уравнение Пуассона):

\begin{equation}
	\label{eqn:source}
	\frac{\partial^2 u}{\partial x^2} + \frac{\partial^2 u}{\partial z^2} + \frac{f(x, z)}{k} = 0.
\end{equation}

На границах прямоугольной области $(0 < x < a, 0 < z < b)$.

\begin{center}
	\scalebox{0.8}{
	\begin{tikzpicture}
		\begin{axis}[
			xlabel={$x$},
			ylabel={$z$},
			every axis x label/.style={at={(current axis.right of origin)},anchor=west},
			every axis y label/.style={at={(current axis.north west)},above=2mm},
			axis lines=left,
			xmin=0,
			xmax=1.1,
			ymin=0,
			ymax=1.25,
			xtick={0, 1},
			xticklabels={$0$, $a$},
			ytick={1},
			yticklabels={$b$},
			tick style={draw=none},
			]
			\addplot[
			mark=none,
			very thick,
			] coordinates{
				(0, 1)
				(1, 1)
			};

			\addplot[
			mark=none,
			very thick,
			] coordinates {
				(1, 0)
				(1, 1)
			};
		\end{axis}
	\end{tikzpicture}
}
\end{center}

Краевые условия:

\begin{equation}
	\begin{dcases}
		x = 0, & u(0, z) = u_0, \\
		x = a, & u(a, z) = u_0, \\
		z = 0, & u(x, 0) = u_0, \\
		z = b, & u(x, b) = u_0. \\
	\end{dcases}
\end{equation}

Значения коэффициентов задачи (все размерности согласованы).

Геометрические размеры $a = b = 10$ см.

$u_0 = 300$ К.

В качестве примера функции источников можно предложить распределение вида $f(x, z) = f_0 e^{\beta(x - a/2)^2(z - b/2)^2}$, параметры $f_0, \beta$ варьируются исходя из условия, чтобы максимум функции не превышал $3000$~K.

Для определённости, положим $f_0 = 100, \beta = -0,0001$ и $k = 2,36$.

\subsection{Физическое содержание задачи}

Сформулированная математическая модель описывает двумерное температурное поле $u(x, z)$ в тонкой прямоугольной пластине с размерами $a \times b$.
Температура по толщине пластины (третьей координате) принимается постоянной.

\section{Конечно-разностная аппроксимация}

Добавим в эллиптическое уравнение координату по времени, чтобы получилось уравнение параболического типа с двумя пространственными переменными:

\begin{equation}
	\frac{\partial u}{\partial t} = \frac{\partial^2 u}{\partial x^2} + \frac{\partial^2 u}{\partial z^2} + \frac{f(x, z)}{k}.
\end{equation}

Для составления схемы, которая носит название продольно-поперченой, введём полуцелый слой $\overline{t} = t_m + \frac\tau2$.
Схема имеет вид

\begin{equation}
	\label{eqn:schema1}
	\frac{\overline y_{ij} - y_{ij}}{0,5\tau} = A_1\overline y_{ij} + A_2y_{ij} + \frac{f_{ij}}k,
\end{equation}

\begin{equation}
	\label{eqn:schema2}
	\frac{\hat y_{ij} - \overline y_{ij}}{0,5\tau} = A_1\overline y_{ij} + A_2\hat y_{ij} + \frac{f_{ij}}k,
\end{equation}

\noindent причём разностные операторы $A_1$, $A_2$ действуют каждый по своему направлению (по своей координате) и определяются выражениями

\begin{equation}
	A_1 y_{ij} = \frac{1}{h_x^2} (y_{i - 1, j} - 2y_{ij} + y_{i + 1, j}),
\end{equation}

\begin{equation}
	A_2 y_{ij} = \frac{1}{h_z^2} (y_{i, j - 1} - 2y_{ij} + y_{i, j + 1}).
\end{equation}

Здесь $1 \leqslant i \leqslant N_x - 1$, $1 \leqslant j \leqslant N_z - 1$.

Схема \eqref{eqn:schema1}, \eqref{eqn:schema2} реализуется следующим образом.
Вначале вычисляют решение на полуцелом слое согласно \eqref{eqn:schema1}.
В системе линейных уравнений \eqref{eqn:schema1} с трёхдиагональной матрицей неизвестными являются величины $\overline y_{ij}$, которые находят прогонкой по индексу $i$ (по координате $x$) для каждого фиксированного значения индекса $j$.
При найденном решении $\overline y_{ij}$ система \eqref{eqn:schema2} также является линейной системой уравнений с трёхдиагональной матрицей, в которой неизвестными выступают $\hat y_{ij}$.
Решение $\hat y_{ij}$ находят прогонкой по индексу $j$ (по координате $z$) для каждого индекса $i$ \cite{1}.

Относительно аппроксимации и устойчивости продольно-поперечной схемы следует отметить, что схема \eqref{eqn:schema1}, \eqref{eqn:schema2} равномерно и безусловно устойчива по начальным данным и по правой части и аппроксимирует задачу на равномерных сетках с погрешностью $O(\tau^2 + h_x^2 + h_z^2)$.

Приведём уравнение \eqref{eqn:schema1} к стандартному виду для прогонки

\begin{equation}
	\frac{2(\overline y_{ij} - y_{ij})}{\tau} = \frac{1}{h_x^2} (\overline y_{i - 1, j} - 2\overline y_{ij} + \overline y_{i + 1, j}) + \frac{1}{h_z^2} (y_{i, j - 1} - 2y_{ij} + y_{i, j + 1}) + \frac{f_{ij}}k,
\end{equation}

\begin{equation}
	\frac{1}{h_x^2}\overline y_{i - 1, j} - 2 \left(\frac1{h_x^2} + \frac1\tau \right)\overline y_{ij} + \frac{1}{h_x^2}\overline y_{i + 1, j}
	= -\frac{1}{h_z^2}y_{i, j - 1} + 2\left(\frac1{h_z^2} - \frac1\tau\right)y_{ij} - \frac{1}{h_z^2}y_{i, j + 1} - \frac{f_{ij}}k,
\end{equation}

\begin{equation}
	\tau h_z^2 \overline y_{i - 1, j} - 2h_z^2(\tau + h_x^2)\overline y_{ij} + \tau h_z^2 \overline y_{i + 1, j}
	= h_x^2 \left( 2(\tau - h_z^2)y_{ij} - \tau \left( y_{i, j - 1} + y_{i, j + 1} + \frac{h_z^2 f_{ij}}k \right) \right)
\end{equation}

Коэффициенты для метода прогонки:

\begin{equation}
	\begin{dcases}
		A_i = \tau h_z^2, \\
		B_i = -2h_z^2(\tau + h_x^2), \\
		C_i = \tau h_z^2, \\
		D_i = h_x^2 \left( 2(\tau - h_z^2)y_{ij} - \tau \left( y_{i, j - 1} + y_{i, j + 1} + \frac{h_z^2 f_{ij}}k \right) \right), \\
		i = \overline{1, N_x - 1}. \\
	\end{dcases}
\end{equation}

Краевые условия:

\begin{equation}
	\begin{aligned}
		A_0     &= 0, & B_0     &= 1, & C_0     &= 0, & D_0     &= u_0, \\
		A_{N_x} &= 0, & B_{N_x} &= 1, & C_{N_x} &= 0, & D_{N_x} &= u_0. \\
	\end{aligned}
\end{equation}


Приведём уравнение \eqref{eqn:schema2} к стандартному виду для прогонки

\begin{equation}
	\frac{2(\hat y_{ij} - \overline y_{ij})}{\tau} = \frac{1}{h_x^2} (\overline y_{i - 1, j} - 2\overline y_{ij} + \overline y_{i + 1, j}) + \frac{1}{h_z^2} (\hat y_{i, j - 1} - 2\hat y_{ij} + \hat y_{i, j + 1}) + \frac{f_{ij}}k,
\end{equation}

\begin{equation}
	\frac1{h_z^2} \hat y_{i, j - 1} - 2 \left(\frac1{h_z^2} + \frac1\tau \right)\hat y_{ij} + \frac1{h_z^2}\hat y_{i, j + 1}
	= -\frac1{h_x^2}\overline y_{i - 1, j} + 2\left(\frac1{h_x^2} - \frac1\tau\right)\overline y_{ij} - \frac1{h_x^2}y_{i + 1, j} - \frac{f_{ij}}k,
\end{equation}

\begin{equation}
	\tau h_x^2 \hat y_{i, j - 1} - 2h_x^2(\tau + h_z^2)\hat y_{ij} + \tau h_x^2 \hat y_{i, j + 1}
	= h_z^2 \left( 2(\tau - h_x^2)\overline y_{ij} - \tau \left( \overline y_{i - 1, j} + \overline y_{i + 1, j} + \frac{h_x^2 f_{ij}}k \right) \right)
\end{equation}

Коэффициенты для метода прогонки:

\begin{equation}
	\begin{dcases}
		A_j = \tau h_x^2, \\
		B_j = -2h_x^2(\tau + h_z^2), \\
		C_j = \tau h_x^2, \\
		D_j = h_z^2 \left( 2(\tau - h_x^2)\overline y_{ij} - \tau \left( \overline y_{i - 1, j} + \overline y_{i + 1, j} + \frac{h_x^2 f_{ij}}k \right) \right), \\
		j = \overline{1, N_z - 1}. \\
	\end{dcases}
\end{equation}

Краевые условия:

\begin{equation}
	\begin{aligned}
		A_0     &= 0, & B_0     &= 1, & C_0     &= 0, & D_0     &= u_0, \\
		A_{N_z} &= 0, & B_{N_z} &= 1, & C_{N_z} &= 0, & D_{N_z} &= u_0. \\
	\end{aligned}
\end{equation}

Линейно проинтерполируем граничные значения $u$ для начальной итерации.
Тогда $y_{i, j}^0 \equiv u_0$.

\section{Вероятностный метод}

Теперь применим статистический метод \cite{2} решения уравнения Пуассона \eqref{eqn:source}.

Покроем область $0 \leqslant x \leqslant a, 0 \leqslant z \leqslant b$ квадратной сеткой с шагом $h$.

Из каждого узла сетки будем моделировать случайное блуждание частиц.
Находясь во внутреннем узле $y_{ij}$, частица $M$ может с равной вероятностью уйти либо влево, либо вправо, либо вверх, либо вниз.
Блуждание частицы $M$ заканчивается, как только она выходит на границу области.

Решение уравнения \eqref{eqn:source} в результате моделирования стохастического блуждания частиц имеет вид:

\begin{equation}
	y(x, z) = \frac{1}{N_p} \sum_{j=1}^{N_p} y_b(j) + \left( \frac{h^2}{4k} \right) \frac{1}{N_p} \sum_{j=1}^{N_p} \overline f_jM_j,
\end{equation}

\noindent где

\begin{itemize}
	\item $N_p$ — количество разыгрываемых частиц из узла;
	\item $y_b(j)$ — граничное значение функции $y$ для $j$-й частицы;
	\item $\overline f_j$ — среднее значение функции $f$ по всем узлам траектории $j$-й частицы;
	\item $M_j$ — количество узлов в траектории $j$-й частицы.
\end{itemize}

\section{Сравнительный анализ методов решения}

Моделирование производилось на компьютере со следующими характеристиками:

\begin{itemize}
	\item ОС: Ubuntu 21.10 64-bit,
	\item процессор: 1,4 ГГц AMD Ryzen 7 5700U,
	\item память: 5,7 ГБ LPDDR4.
\end{itemize}

Сравним результаты моделирования на сетке с шагом $h_x = h_z = h = 1$ см, $\tau = 1$~с, $\varepsilon = 0,0001$, $N_p = 5000$.
В таблицах \ref{tbl:adi}—\ref{tbl:probability} представлены температурные поля, полученные продольно-поперечным и вероятностным методами соответственно.
В таблице \ref{tbl:diff} представлена разница значений из таблиц \ref{tbl:adi} и \ref{tbl:probability}.

\begin{table}[ht]
	\small
	\caption{Температурное поле, полученное продольно-поперечным методом}
	\label{tbl:adi}
	\begin{tabular}{|l|l|l|l|l|l|l|l|l|l|l|}
		\hline
		300 & 300     & 300     & 300     & 300     & 300     & 300     & 300     & 300     & 300     & 300 \\ \hline
		300 & 349.322 & 382.246 & 403.667 & 415.839 & 419.795 & 415.839 & 403.667 & 382.246 & 349.322 & 300 \\ \hline
		300 & 382.246 & 439.312 & 476.848 & 498.206 & 505.144 & 498.206 & 476.848 & 439.312 & 382.246 & 300 \\ \hline
		300 & 403.667 & 476.848 & 525.356 & 553.028 & 562.022 & 553.028 & 525.356 & 476.848 & 403.667 & 300 \\ \hline
		300 & 415.839 & 498.206 & 553.028 & 584.362 & 594.552 & 584.362 & 553.028 & 498.206 & 415.839 & 300 \\ \hline
		300 & 419.795 & 505.144 & 562.022 & 594.552 & 605.133 & 594.552 & 562.022 & 505.144 & 419.795 & 300 \\ \hline
		300 & 415.839 & 498.206 & 553.028 & 584.362 & 594.552 & 584.362 & 553.028 & 498.206 & 415.839 & 300 \\ \hline
		300 & 403.667 & 476.848 & 525.356 & 553.028 & 562.022 & 553.028 & 525.356 & 476.848 & 403.667 & 300 \\ \hline
		300 & 382.246 & 439.312 & 476.848 & 498.206 & 505.144 & 498.206 & 476.848 & 439.312 & 382.246 & 300 \\ \hline
		300 & 349.322 & 382.246 & 403.667 & 415.839 & 419.795 & 415.839 & 403.667 & 382.246 & 349.322 & 300 \\ \hline
		300 & 300     & 300     & 300     & 300     & 300     & 300     & 300     & 300     & 300     & 300 \\ \hline
	\end{tabular}
\end{table}

\begin{table}[ht]
	\small
	\caption{Температурное поле, полученное вероятностным методом}
	\label{tbl:probability}
	\begin{tabular}{|l|l|l|l|l|l|l|l|l|l|l|}
		\hline
		300 & 300     & 300     & 300     & 300     & 300     & 300     & 300     & 300     & 300     & 300 \\ \hline
		300 & 347.997 & 382.004 & 404.381 & 416.424 & 418.476 & 414.192 & 402.982 & 381.346 & 349.058 & 300 \\ \hline
		300 & 380.461 & 438.8   & 476.483 & 498.021 & 504.324 & 497.636 & 475.434 & 438.4   & 381.316 & 300 \\ \hline
		300 & 403.046 & 476.336 & 524.314 & 552.303 & 560.19  & 551.834 & 524.416 & 476.116 & 404.203 & 300 \\ \hline
		300 & 416.413 & 496.967 & 551.949 & 583.623 & 593.486 & 583.306 & 551.52  & 496.718 & 414.992 & 300 \\ \hline
		300 & 419.923 & 505.379 & 561.016 & 594.109 & 604.455 & 594.093 & 561.223 & 503.058 & 417.508 & 300 \\ \hline
		300 & 415.592 & 497.442 & 552.094 & 583.408 & 593.941 & 583.008 & 552.044 & 497.394 & 415.749 & 300 \\ \hline
		300 & 404.608 & 476.058 & 524.999 & 551.775 & 561.399 & 552.58  & 524.523 & 476.29  & 403.891 & 300 \\ \hline
		300 & 380.683 & 437.926 & 475.802 & 497.039 & 503.002 & 497.552 & 477.432 & 438.181 & 381.123 & 300 \\ \hline
		300 & 349.045 & 381.228 & 403.179 & 414.452 & 418.306 & 415.365 & 404.663 & 380.41  & 348.885 & 300 \\ \hline
		300 & 300     & 300     & 300     & 300     & 300     & 300     & 300     & 300     & 300     & 300 \\ \hline
	\end{tabular}
\end{table}

\begin{table}[H]
	\small
	\caption{Поэлементная разница таблиц \ref{tbl:adi} и \ref{tbl:probability}}
	\label{tbl:diff}
	\begin{tabular}{|l|l|l|l|l|l|l|l|l|l|l|}
		\hline
		0 & 0       & 0       & 0       & 0       & 0      & 0      & 0       & 0      & 0       & 0 \\ \hline
		0 & 1.3251  & 0.2412  & -0.7148 & -0.5854 & 1.3184 & 1.6466 & 0.6846  & 0.9001 & 0.2646  & 0 \\ \hline
		0 & 1.7847  & 0.5121  & 0.3652  & 0.1857  & 0.8197 & 0.5709 & 1.4147  & 0.9124 & 0.9296  & 0 \\ \hline
		0 & 0.6211  & 0.5126  & 1.0419  & 0.7250  & 1.8322 & 1.1945 & 0.9396  & 0.7325 & -0.5368 & 0 \\ \hline
		0 & -0.5746 & 1.2392  & 1.0793  & 0.7389  & 1.0660 & 1.0558 & 1.5089  & 1.4886 & 0.8467  & 0 \\ \hline
		0 & -0.1284 & -0.2355 & 1.0058  & 0.4429  & 0.6780 & 0.4594 & 0.7987  & 2.0860 & 2.2866  & 0 \\ \hline
		0 & 0.2469  & 0.7649  & 0.9349  & 0.9543  & 0.6106 & 1.3542 & 0.9847  & 0.8122 & 0.0897  & 0 \\ \hline
		0 & -0.9411 & 0.7900  & 0.3566  & 1.2536  & 0.6229 & 0.4485 & 0.8331  & 0.5580 & -0.2247 & 0 \\ \hline
		0 & 1.5626  & 1.3856  & 1.0466  & 1.1677  & 2.1420 & 0.6543 & -0.5841 & 1.1308 & 1.1225  & 0 \\ \hline
		0 & 0.2772  & 1.0171  & 0.4881  & 1.3865  & 1.4886 & 0.4736 & -0.9966 & 1.8360 & 0.4370  & 0 \\ \hline
		0 & 0       & 0       & 0       & 0       & 0      & 0      & 0       & 0      & 0       & 0 \\ \hline
	\end{tabular}
\end{table}

Наибольшая разница в таблице \ref{tbl:diff} по модулю не превосходит $2.2866$.
Таким образом, можно заключить, что программная реализация методов даёт приблизительно одинаковый результат.

Для $h = 1$ см конечно-разностный алгоритм отработал за $0,17$ мс, а вероятностный — за $220,45$ мс.
Такое время вероятностного метода сложно назвать приемлемым для столь большого шага. Можно предложить ряд усовершенствований стохастического алгоритма.

\begin{enumerate}
	\item Введём следующее понятие: под $i$-м слоем прямоугольной сетки будем понимать все узлы сетки, у которых минимальное расстояние до границы равно $i$. Таким образом, нулевой слой это и есть граница, первый слой — граница сетки, если из неё исключить нулевой слой, и т. д.
	Тогда исходный алгоритм можно улучшить, если начинать розыгрыш частиц с первого слоя, затем со второго и так до последнего, при этом, как только все узлы из $i$-го слоя будут посчитаны, для $i+1$ слоя примем $i$ как границу. Таким образом, область будет сужаться, время моделирования — уменьшаться.
	\item Для внутренних слоёв можно уменьшить количество разыгрываемых частиц, т. к. граничными для них являются предыдущие слои, на которых значение $y(x, z)$ уже было посчитано. Для $i$-го слоя будем разыгрывать $N_p / i$ частиц.
	\item Распараллелим вычисления. Траектории пучка частиц, выпущенных из текущего узла, друг от друга не зависят.
\end{enumerate}

В таблице \ref{tbl:time} представлено время работы методов в зависимости от некоторых значений шага сетки $h$.
Столбец \textbf{1} в таблице обозначает исходный стохастический алгоритм, столбцы \textbf{2}—\textbf{4} соответствуют вышеназванные улучшения.

\begin{table}[H]
	\small
	\caption{Время работы методов}
	\label{tbl:time}
	\begin{tabular}{|l|lllll|}
		\hline
		\multicolumn{1}{|c|}{\multirow{3}{*}{\textbf{h, см}}} & \multicolumn{5}{c|}{\textbf{Время, мс}}                                                                                                                                                                     \\ \cline{2-6}
		\multicolumn{1}{|c|}{}                                & \multicolumn{1}{c|}{\multirow{2}{*}{\textbf{Продольно-поперечный}}} & \multicolumn{4}{c|}{\textbf{Вероятностный}}                                                                                           \\ \cline{3-6}
		\multicolumn{1}{|c|}{}                                & \multicolumn{1}{c|}{}                                               & \multicolumn{1}{c|}{\textbf{1}} & \multicolumn{1}{c|}{\textbf{2}} & \multicolumn{1}{c|}{\textbf{3}} & \multicolumn{1}{c|}{\textbf{4}} \\ \hline
		1                                                     & \multicolumn{1}{l|}{0.17}                                           & \multicolumn{1}{l|}{220.45}     & \multicolumn{1}{l|}{101.15}     & \multicolumn{1}{l|}{71.56}      & 45.59                           \\ \hline
		1/2                                                   & \multicolumn{1}{l|}{0.52}                                           & \multicolumn{1}{l|}{2'990}      & \multicolumn{1}{l|}{769}        & \multicolumn{1}{l|}{379}        & 201                             \\ \hline
		1/3                                                   & \multicolumn{1}{l|}{1.08}                                           & \multicolumn{1}{l|}{14'615}     & \multicolumn{1}{l|}{2'515}      & \multicolumn{1}{l|}{987}        & 466                             \\ \hline
		1/4                                                   & \multicolumn{1}{l|}{1.80}                                           & \multicolumn{1}{l|}{45'529}     & \multicolumn{1}{l|}{5'793}      & \multicolumn{1}{l|}{1'937}      & 837                             \\ \hline
		1/5                                                   & \multicolumn{1}{l|}{2.74}                                           & \multicolumn{1}{l|}{110'545}    & \multicolumn{1}{l|}{11'111}     & \multicolumn{1}{l|}{3'257}      & 1'321                           \\ \hline
		1/10                                                  & \multicolumn{1}{l|}{9.88}                                           & \multicolumn{1}{l|}{-}          & \multicolumn{1}{l|}{85'994}     & \multicolumn{1}{l|}{15'781}     & 5'755                           \\ \hline
		1/20                                                  & \multicolumn{1}{l|}{45.22}                                          & \multicolumn{1}{l|}{-}          & \multicolumn{1}{l|}{-}          & \multicolumn{1}{l|}{-}          & 21'950                          \\ \hline
		1/50                                                  & \multicolumn{1}{l|}{4'605}                                          & \multicolumn{1}{l|}{-}          & \multicolumn{1}{l|}{-}          & \multicolumn{1}{l|}{-}          & -                               \\ \hline
	\end{tabular}
\end{table}

На рисунке \ref{img:viz} представлена визуализация данных из таблицы \ref{tbl:time}.·

\begin{figure}[ht]
	\begin{center}
		\begin{tikzpicture}
			\begin{axis}[
				xlabel={$N$},
				ylabel={Время, мс},
				xmin=5, xmax=55,
				ymin=0, ymax=12000,
				legend pos=north west,
				ymajorgrids=true,
				grid style=dashed,
				]

				\addplot[color=blue, mark=oplus*]
				coordinates {
					(11, 0.17) (21, 0.52) (31, 1.08) (41, 1.80) (51, 2.74)
				};
				\addplot[color=red, mark=square]
				coordinates {
					(11, 220) (21, 2990) (31, 14615) (41, 45529) %(51, 110545)
				};
				\addplot[color=black, mark=square*]
				coordinates {
					(11, 101) (21, 769) (31, 2515) (41, 5793) (51, 11111)
				};
				\addplot[color=brown, mark=o]
				coordinates {
					(11, 71.56) (21, 379) (31, 987) (41, 1937) (51, 3257)
				};
				\addplot[color=magenta, mark=triangle*]
				coordinates {
					(11, 45.59) (21, 201) (31, 466) (41, 837) (51, 1321)
				};
				\legend{ПП, В1, В2, В3, В4}
			\end{axis}
		\end{tikzpicture}
		\captionsetup{justification=centering}
		\caption{Зависимость времени работы методов от $N$}
		\label{img:viz}
	\end{center}
\end{figure}

Таким образом, по результатам моделирования видно, что алгоритм, основанный на конечно-разностной аппроксимации дифференциального уравнения \eqref{eqn:source}, является более быстрым по времени по сравнению с вероятностным методом. Предложенные улучшения стохастического метода уменьшили время моделирования более, чем в $4N$ раз.

\section*{Заключение}

В ходе численного решения стационарного многомерного эллиптического уравнения теплопроводности выведены математические соотношения конечно-разностной аппроксимации (коэффициенты для метода матричной прогонки решения продольно-поперечной схемы) и вероятностного метода. Проведён анализ программных реализаций методов на быстродействие. Выяснено, что наибольший интерес в плане производительности представляет конечно-разностная аппроксимация. Был предложен ряд улучшений вероятностного метода, уменьшивший время моделирования более, чем в $4N$ раз, т. е. асимптотическая сложность алгоритма понижена на порядок.

\begin{thebibliography}{5}
	\bibitem{1} Градов В. М. Курс лекций по моделированию. МГТУ им. Н. Э. Баумана. — 2020
	\bibitem{2} Кузнецов В. Ф. Решение задач теплопроводности методом Монте-Карло. — 1973
\end{thebibliography}

\noindent \textbf{Керимов Ахмед Шахович} — студент, МГТУ им. Н. Э. Баумана, кафедра «Программное обеспечение ЭВМ и информационные технологии».


\end{document}
