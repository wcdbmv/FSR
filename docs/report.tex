\documentclass[12pt, a4paper]{article}

\include{preamble}

\begin{document}


\noindent УДК~004.94

\hfill

\noindent \textbf{Сравнительный анализ разностного и вероятностного методов исследования математической модели, построенной на дифференциальном уравнении в частных производных эллиптического типа}

\noindent А.~Ш.~Керимов$^{1}$\hfill kerimov.edu@yandex.ru

\noindent $^{1}$МГТУ им.~Н.~Э.~Баумана, Москва, Россия

\hfill

\noindent \textbf{Аннотация}

\noindent Статья посвящена численным методам приближённого решения стационарного многомерного эллиптического уравнения теплопроводности.
Описаны математические соотношения конечно-разностного и вероятностного методов. Проведён анализ программных реализаций методов на быстродействие. Предложены алгоритмические улучшения вероятностного метода.

\noindent \textbf{Ключевые слова}

\noindent \textit{математическая физика, задача двумерной теплопроводности, численные методы, уравнение Пуассона, продольно-поперечная схема, матричная прогонка, вероятностный метод, метод Монте-Карло}

\hfill

\section*{Введение}

Прикладные проблемы приводят к необходимости решения краевых задач для уравнений с частными производными. Разработка приближённых методов их решения базируется на построении и исследовании численных методов решения краевых задач для базовых (основных, модельных) уравнений математической физики. В качестве таковых при рассмотрении уравнений второго порядка выделяются эллиптические, параболические и гиперболические уравнения.

\section{Постановка задачи}

Задана математическая модель с постоянными коэффициентами $k(x, z) \equiv k$ (уравнение Пуассона):

\begin{equation}
	\label{eqn:source}
	\frac{\partial^2 u}{\partial x^2} + \frac{\partial^2 u}{\partial z^2} + \frac{f(x, z)}{k} = 0.
\end{equation}

На границах прямоугольной области $(0 < x < a, 0 < z < b)$.

\begin{center}
	\scalebox{0.8}{
	\begin{tikzpicture}
		\begin{axis}[
			xlabel={$x$},
			ylabel={$z$},
			every axis x label/.style={at={(current axis.right of origin)},anchor=west},
			every axis y label/.style={at={(current axis.north west)},above=2mm},
			axis lines=left,
			xmin=0,
			xmax=1.1,
			ymin=0,
			ymax=1.25,
			xtick={0, 1},
			xticklabels={$0$, $a$},
			ytick={1},
			yticklabels={$b$},
			tick style={draw=none},
			]
			\addplot[
			mark=none,
			very thick,
			] coordinates{
				(0, 1)
				(1, 1)
			};

			\addplot[
			mark=none,
			very thick,
			] coordinates {
				(1, 0)
				(1, 1)
			};
		\end{axis}
	\end{tikzpicture}
}
\end{center}

Краевые условия:

\begin{equation}
	\begin{dcases}
		x = 0, & u(0, z) = u_0, \\
		x = a, & u(a, z) = u_0, \\
		z = 0, & u(x, 0) = u_0, \\
		z = b, & u(x, b) = u_0. \\
	\end{dcases}
\end{equation}

Значения коэффициентов задачи (все размерности согласованы).

Геометрические размеры $a = b = 10$ см.

$u_0 = 300$ К.

В качестве примера функции источников можно предложить распределение вида $f(x, z) = f_0 e^{\beta(x - a/2)^2(z - b/2)^2}$, параметры $f_0, \beta$ варьируются исходя из условия, чтобы максимум функции не превышал $3000$~K.

Для определённости, положим $f_0 = 100, \beta = -0,0001$ и $k = 2,36$.

\subsection{Физическое содержание задачи}

Сформулированная математическая модель описывает двумерное температурное поле $u(x, z)$ в тонкой прямоугольной пластине с размерами $a \times b$.
Температура по толщине пластины (третьей координате) принимается постоянной.

\section{Конечно-разностная аппроксимация}

Добавим в эллиптическое уравнение координату по времени, чтобы получилось уравнение параболического типа с двумя пространственными переменными:

\begin{equation}
	\frac{\partial u}{\partial t} = \frac{\partial^2 u}{\partial x^2} + \frac{\partial^2 u}{\partial z^2} + \frac{f(x, z)}{k}.
\end{equation}

Для составления схемы, которая носит название продольно-поперченой, введём полуцелый слой $\overline{t} = t_m + \frac\tau2$.
Схема имеет вид

\begin{equation}
	\label{eqn:schema1}
	\frac{\overline y_{ij} - y_{ij}}{0,5\tau} = A_1\overline y_{ij} + A_2y_{ij} + \frac{f_{ij}}k,
\end{equation}

\begin{equation}
	\label{eqn:schema2}
	\frac{\hat y_{ij} - \overline y_{ij}}{0,5\tau} = A_1\overline y_{ij} + A_2\hat y_{ij} + \frac{f_{ij}}k,
\end{equation}

\noindent причём разностные операторы $A_1$, $A_2$ действуют каждый по своему направлению (по своей координате) и определяются выражениями

\begin{equation}
	A_1 y_{ij} = \frac{1}{h_x^2} (y_{i - 1, j} - 2y_{ij} + y_{i + 1, j}),
\end{equation}

\begin{equation}
	A_2 y_{ij} = \frac{1}{h_z^2} (y_{i, j - 1} - 2y_{ij} + y_{i, j + 1}).
\end{equation}

Здесь $1 \leqslant i \leqslant N_x - 1$, $1 \leqslant j \leqslant N_z - 1$.

Схема \eqref{eqn:schema1}, \eqref{eqn:schema2} реализуется следующим образом.
Вначале вычисляют решение на полуцелом слое согласно \eqref{eqn:schema1}.
В системе линейных уравнений \eqref{eqn:schema1} с трёхдиагональной матрицей неизвестными являются величины $\overline y_{ij}$, которые находят прогонкой по индексу $i$ (по координате $x$) для каждого фиксированного значения индекса $j$.
При найденном решении $\overline y_{ij}$ система \eqref{eqn:schema2} также является линейной системой уравнений с трёхдиагональной матрицей, в которой неизвестными выступают $\hat y_{ij}$.
Решение $\hat y_{ij}$ находят прогонкой по индексу $j$ (по координате $z$) для каждого индекса $i$ \cite{1}.

Относительно аппроксимации и устойчивости продольно-поперечной схемы следует отметить, что схема \eqref{eqn:schema1}, \eqref{eqn:schema2} равномерно и безусловно устойчива по начальным данным и по правой части и аппроксимирует задачу на равномерных сетках с погрешностью $O(\tau^2 + h_x^2 + h_z^2)$.

Приведём уравнение \eqref{eqn:schema1} к стандартному виду для прогонки

\begin{equation}
	\frac{2(\overline y_{ij} - y_{ij})}{\tau} = \frac{1}{h_x^2} (\overline y_{i - 1, j} - 2\overline y_{ij} + \overline y_{i + 1, j}) + \frac{1}{h_z^2} (y_{i, j - 1} - 2y_{ij} + y_{i, j + 1}) + \frac{f_{ij}}k,
\end{equation}

\begin{equation}
	\frac{1}{h_x^2}\overline y_{i - 1, j} - 2 \left(\frac1{h_x^2} + \frac1\tau \right)\overline y_{ij} + \frac{1}{h_x^2}\overline y_{i + 1, j}
	= -\frac{1}{h_z^2}y_{i, j - 1} + 2\left(\frac1{h_z^2} - \frac1\tau\right)y_{ij} - \frac{1}{h_z^2}y_{i, j + 1} - \frac{f_{ij}}k,
\end{equation}

\begin{equation}
	\tau h_z^2 \overline y_{i - 1, j} - 2h_z^2(\tau + h_x^2)\overline y_{ij} + \tau h_z^2 \overline y_{i + 1, j}
	= h_x^2 \left( 2(\tau - h_z^2)y_{ij} - \tau \left( y_{i, j - 1} + y_{i, j + 1} + \frac{h_z^2 f_{ij}}k \right) \right)
\end{equation}

Коэффициенты для метода прогонки:

\begin{equation}
	\begin{dcases}
		A_i = \tau h_z^2, \\
		B_i = -2h_z^2(\tau + h_x^2), \\
		C_i = \tau h_z^2, \\
		D_i = h_x^2 \left( 2(\tau - h_z^2)y_{ij} - \tau \left( y_{i, j - 1} + y_{i, j + 1} + \frac{h_z^2 f_{ij}}k \right) \right), \\
		i = \overline{1, N_x - 1}. \\
	\end{dcases}
\end{equation}

Краевые условия:

\begin{equation}
	\begin{aligned}
		A_0     &= 0, & B_0     &= 1, & C_0     &= 0, & D_0     &= u_0, \\
		A_{N_x} &= 0, & B_{N_x} &= 1, & C_{N_x} &= 0, & D_{N_x} &= u_0. \\
	\end{aligned}
\end{equation}


Приведём уравнение \eqref{eqn:schema2} к стандартному виду для прогонки

\begin{equation}
	\frac{2(\hat y_{ij} - \overline y_{ij})}{\tau} = \frac{1}{h_x^2} (\overline y_{i - 1, j} - 2\overline y_{ij} + \overline y_{i + 1, j}) + \frac{1}{h_z^2} (\hat y_{i, j - 1} - 2\hat y_{ij} + \hat y_{i, j + 1}) + \frac{f_{ij}}k,
\end{equation}

\begin{equation}
	\frac1{h_z^2} \hat y_{i, j - 1} - 2 \left(\frac1{h_z^2} + \frac1\tau \right)\hat y_{ij} + \frac1{h_z^2}\hat y_{i, j + 1}
	= -\frac1{h_x^2}\overline y_{i - 1, j} + 2\left(\frac1{h_x^2} - \frac1\tau\right)\overline y_{ij} - \frac1{h_x^2}y_{i + 1, j} - \frac{f_{ij}}k,
\end{equation}

\begin{equation}
	\tau h_x^2 \hat y_{i, j - 1} - 2h_x^2(\tau + h_z^2)\hat y_{ij} + \tau h_x^2 \hat y_{i, j + 1}
	= h_z^2 \left( 2(\tau - h_x^2)\overline y_{ij} - \tau \left( \overline y_{i - 1, j} + \overline y_{i + 1, j} + \frac{h_x^2 f_{ij}}k \right) \right)
\end{equation}

Коэффициенты для метода прогонки:

\begin{equation}
	\begin{dcases}
		A_j = \tau h_x^2, \\
		B_j = -2h_x^2(\tau + h_z^2), \\
		C_j = \tau h_x^2, \\
		D_j = h_z^2 \left( 2(\tau - h_x^2)\overline y_{ij} - \tau \left( \overline y_{i - 1, j} + \overline y_{i + 1, j} + \frac{h_x^2 f_{ij}}k \right) \right), \\
		j = \overline{1, N_z - 1}. \\
	\end{dcases}
\end{equation}

Краевые условия:

\begin{equation}
	\begin{aligned}
		A_0     &= 0, & B_0     &= 1, & C_0     &= 0, & D_0     &= u_0, \\
		A_{N_z} &= 0, & B_{N_z} &= 1, & C_{N_z} &= 0, & D_{N_z} &= u_0. \\
	\end{aligned}
\end{equation}

Линейно проинтерполируем граничные значения $u$ для начальной итерации.
Тогда $y_{i, j}^0 \equiv u_0$.

\section{Вероятностный метод}

Теперь применим статистический метод \cite{2} решения уравнения Пуассона \eqref{eqn:source}.

Покроем область $0 \leqslant x \leqslant a, 0 \leqslant z \leqslant b$ квадратной сеткой с шагом $h$.

Из каждого узла сетки будем моделировать случайное блуждание частиц.
Находясь во внутреннем узле $y_{ij}$, частица $M$ может с равной вероятностью уйти либо влево, либо вправо, либо вверх, либо вниз.
Блуждание частицы $M$ заканчивается, как только она выходит на границу области.

Решение уравнения \eqref{eqn:source} в результате моделирования стохастического блуждания частиц имеет вид:

\begin{equation}
	y(x, z) = \frac{1}{N_p} \sum_{j=1}^{N_p} y_b(j) + \left( \frac{h^2}{4k} \right) \frac{1}{N_p} \sum_{j=1}^{N_p} \overline f_jM_j,
\end{equation}

\noindent где

\begin{itemize}
	\item $N_p$ — количество разыгрываемых частиц из узла;
	\item $y_b(j)$ — граничное значение функции $y$ для $j$-й частицы;
	\item $\overline f_j$ — среднее значение функции $f$ по всем узлам траектории $j$-й частицы;
	\item $M_j$ — количество узлов в траектории $j$-й частицы.
\end{itemize}

\section{Сравнительный анализ методов решения}

Моделирование производилось на компьютере со следующими характеристиками:

\begin{itemize}
	\item ОС: Ubuntu 21.10 64-bit,
	\item процессор: 1,4 ГГц AMD Ryzen 7 5700U,
	\item память: 5,7 ГБ LPDDR4.
\end{itemize}

Сравним результаты моделирования на сетке с шагом $h_x = h_z = h = 1$ см, $\tau = 1$~с, $\varepsilon = 0,0001$, $N_p = 5000$.
В таблицах \ref{tbl:adi}—\ref{tbl:probability} представлены температурные поля, полученные продольно-поперечным и вероятностным методами соответственно.
В таблице \ref{tbl:diff} представлена разница значений из таблиц \ref{tbl:adi} и \ref{tbl:probability}.

\begin{table}[ht]
	\small
	\caption{Температурное поле, полученное продольно-поперечным методом}
	\label{tbl:adi}
	\begin{tabular}{|l|l|l|l|l|l|l|l|l|l|l|}
		\hline
		300 & 300     & 300     & 300     & 300     & 300     & 300     & 300     & 300     & 300     & 300 \\ \hline
		300 & 349.322 & 382.246 & 403.667 & 415.839 & 419.795 & 415.839 & 403.667 & 382.246 & 349.322 & 300 \\ \hline
		300 & 382.246 & 439.312 & 476.848 & 498.206 & 505.144 & 498.206 & 476.848 & 439.312 & 382.246 & 300 \\ \hline
		300 & 403.667 & 476.848 & 525.356 & 553.028 & 562.022 & 553.028 & 525.356 & 476.848 & 403.667 & 300 \\ \hline
		300 & 415.839 & 498.206 & 553.028 & 584.362 & 594.552 & 584.362 & 553.028 & 498.206 & 415.839 & 300 \\ \hline
		300 & 419.795 & 505.144 & 562.022 & 594.552 & 605.133 & 594.552 & 562.022 & 505.144 & 419.795 & 300 \\ \hline
		300 & 415.839 & 498.206 & 553.028 & 584.362 & 594.552 & 584.362 & 553.028 & 498.206 & 415.839 & 300 \\ \hline
		300 & 403.667 & 476.848 & 525.356 & 553.028 & 562.022 & 553.028 & 525.356 & 476.848 & 403.667 & 300 \\ \hline
		300 & 382.246 & 439.312 & 476.848 & 498.206 & 505.144 & 498.206 & 476.848 & 439.312 & 382.246 & 300 \\ \hline
		300 & 349.322 & 382.246 & 403.667 & 415.839 & 419.795 & 415.839 & 403.667 & 382.246 & 349.322 & 300 \\ \hline
		300 & 300     & 300     & 300     & 300     & 300     & 300     & 300     & 300     & 300     & 300 \\ \hline
	\end{tabular}
\end{table}

\begin{table}[ht]
	\small
	\caption{Температурное поле, полученное вероятностным методом}
	\label{tbl:probability}
	\begin{tabular}{|l|l|l|l|l|l|l|l|l|l|l|}
		\hline
		300 & 300     & 300     & 300     & 300     & 300     & 300     & 300     & 300     & 300     & 300 \\ \hline
		300 & 347.997 & 382.004 & 404.381 & 416.424 & 418.476 & 414.192 & 402.982 & 381.346 & 349.058 & 300 \\ \hline
		300 & 380.461 & 438.8   & 476.483 & 498.021 & 504.324 & 497.636 & 475.434 & 438.4   & 381.316 & 300 \\ \hline
		300 & 403.046 & 476.336 & 524.314 & 552.303 & 560.19  & 551.834 & 524.416 & 476.116 & 404.203 & 300 \\ \hline
		300 & 416.413 & 496.967 & 551.949 & 583.623 & 593.486 & 583.306 & 551.52  & 496.718 & 414.992 & 300 \\ \hline
		300 & 419.923 & 505.379 & 561.016 & 594.109 & 604.455 & 594.093 & 561.223 & 503.058 & 417.508 & 300 \\ \hline
		300 & 415.592 & 497.442 & 552.094 & 583.408 & 593.941 & 583.008 & 552.044 & 497.394 & 415.749 & 300 \\ \hline
		300 & 404.608 & 476.058 & 524.999 & 551.775 & 561.399 & 552.58  & 524.523 & 476.29  & 403.891 & 300 \\ \hline
		300 & 380.683 & 437.926 & 475.802 & 497.039 & 503.002 & 497.552 & 477.432 & 438.181 & 381.123 & 300 \\ \hline
		300 & 349.045 & 381.228 & 403.179 & 414.452 & 418.306 & 415.365 & 404.663 & 380.41  & 348.885 & 300 \\ \hline
		300 & 300     & 300     & 300     & 300     & 300     & 300     & 300     & 300     & 300     & 300 \\ \hline
	\end{tabular}
\end{table}

\begin{table}[H]
	\small
	\caption{Поэлементная разница таблиц \ref{tbl:adi} и \ref{tbl:probability}}
	\label{tbl:diff}
	\begin{tabular}{|l|l|l|l|l|l|l|l|l|l|l|}
		\hline
		0 & 0       & 0       & 0       & 0       & 0      & 0      & 0       & 0      & 0       & 0 \\ \hline
		0 & 1.3251  & 0.2412  & -0.7148 & -0.5854 & 1.3184 & 1.6466 & 0.6846  & 0.9001 & 0.2646  & 0 \\ \hline
		0 & 1.7847  & 0.5121  & 0.3652  & 0.1857  & 0.8197 & 0.5709 & 1.4147  & 0.9124 & 0.9296  & 0 \\ \hline
		0 & 0.6211  & 0.5126  & 1.0419  & 0.7250  & 1.8322 & 1.1945 & 0.9396  & 0.7325 & -0.5368 & 0 \\ \hline
		0 & -0.5746 & 1.2392  & 1.0793  & 0.7389  & 1.0660 & 1.0558 & 1.5089  & 1.4886 & 0.8467  & 0 \\ \hline
		0 & -0.1284 & -0.2355 & 1.0058  & 0.4429  & 0.6780 & 0.4594 & 0.7987  & 2.0860 & 2.2866  & 0 \\ \hline
		0 & 0.2469  & 0.7649  & 0.9349  & 0.9543  & 0.6106 & 1.3542 & 0.9847  & 0.8122 & 0.0897  & 0 \\ \hline
		0 & -0.9411 & 0.7900  & 0.3566  & 1.2536  & 0.6229 & 0.4485 & 0.8331  & 0.5580 & -0.2247 & 0 \\ \hline
		0 & 1.5626  & 1.3856  & 1.0466  & 1.1677  & 2.1420 & 0.6543 & -0.5841 & 1.1308 & 1.1225  & 0 \\ \hline
		0 & 0.2772  & 1.0171  & 0.4881  & 1.3865  & 1.4886 & 0.4736 & -0.9966 & 1.8360 & 0.4370  & 0 \\ \hline
		0 & 0       & 0       & 0       & 0       & 0      & 0      & 0       & 0      & 0       & 0 \\ \hline
	\end{tabular}
\end{table}

Наибольшая разница в таблице \ref{tbl:diff} по модулю не превосходит $2.2866$.
Таким образом, можно заключить, что программная реализация методов даёт приблизительно одинаковый результат.

Для $h = 1$ см конечно-разностный алгоритм отработал за $0,17$ мс, а вероятностный — за $220,45$ мс.
Такое время вероятностного метода сложно назвать приемлемым для столь большого шага. Можно предложить ряд усовершенствований стохастического алгоритма.

\begin{enumerate}
	\item Введём следующее понятие: под $i$-м слоем прямоугольной сетки будем понимать все узлы сетки, у которых минимальное расстояние до границы равно $i$. Таким образом, нулевой слой это и есть граница, первый слой — граница сетки, если из неё исключить нулевой слой, и т. д.
	Тогда исходный алгоритм можно улучшить, если начинать розыгрыш частиц с первого слоя, затем со второго и так до последнего, при этом, как только все узлы из $i$-го слоя будут посчитаны, для $i+1$ слоя примем $i$ как границу. Таким образом, область будет сужаться, время моделирования — уменьшаться.
	\item Для внутренних слоёв можно уменьшить количество разыгрываемых частиц, т. к. граничными для них являются предыдущие слои, на которых значение $y(x, z)$ уже было посчитано. Для $i$-го слоя будем разыгрывать $N_p / i$ частиц.
	\item Распараллелим вычисления. Траектории пучка частиц, выпущенных из текущего узла, друг от друга не зависят.
\end{enumerate}

В таблице \ref{tbl:time} представлено время работы методов в зависимости от некоторых значений шага сетки $h$.
Столбец \textbf{1} в таблице обозначает исходный стохастический алгоритм, столбцы \textbf{2}—\textbf{4} соответствуют вышеназванные улучшения.

\begin{table}[H]
	\small
	\caption{Время работы методов}
	\label{tbl:time}
	\begin{tabular}{|l|lllll|}
		\hline
		\multicolumn{1}{|c|}{\multirow{3}{*}{\textbf{h, см}}} & \multicolumn{5}{c|}{\textbf{Время, мс}}                                                                                                                                                                     \\ \cline{2-6}
		\multicolumn{1}{|c|}{}                                & \multicolumn{1}{c|}{\multirow{2}{*}{\textbf{Продольно-поперечный}}} & \multicolumn{4}{c|}{\textbf{Вероятностный}}                                                                                           \\ \cline{3-6}
		\multicolumn{1}{|c|}{}                                & \multicolumn{1}{c|}{}                                               & \multicolumn{1}{c|}{\textbf{1}} & \multicolumn{1}{c|}{\textbf{2}} & \multicolumn{1}{c|}{\textbf{3}} & \multicolumn{1}{c|}{\textbf{4}} \\ \hline
		1                                                     & \multicolumn{1}{l|}{0.17}                                           & \multicolumn{1}{l|}{220.45}     & \multicolumn{1}{l|}{101.15}     & \multicolumn{1}{l|}{71.56}      & 45.59                           \\ \hline
		1/2                                                   & \multicolumn{1}{l|}{0.52}                                           & \multicolumn{1}{l|}{2'990}      & \multicolumn{1}{l|}{769}        & \multicolumn{1}{l|}{379}        & 201                             \\ \hline
		1/3                                                   & \multicolumn{1}{l|}{1.08}                                           & \multicolumn{1}{l|}{14'615}     & \multicolumn{1}{l|}{2'515}      & \multicolumn{1}{l|}{987}        & 466                             \\ \hline
		1/4                                                   & \multicolumn{1}{l|}{1.80}                                           & \multicolumn{1}{l|}{45'529}     & \multicolumn{1}{l|}{5'793}      & \multicolumn{1}{l|}{1'937}      & 837                             \\ \hline
		1/5                                                   & \multicolumn{1}{l|}{2.74}                                           & \multicolumn{1}{l|}{110'545}    & \multicolumn{1}{l|}{11'111}     & \multicolumn{1}{l|}{3'257}      & 1'321                           \\ \hline
		1/10                                                  & \multicolumn{1}{l|}{9.88}                                           & \multicolumn{1}{l|}{-}          & \multicolumn{1}{l|}{85'994}     & \multicolumn{1}{l|}{15'781}     & 5'755                           \\ \hline
		1/20                                                  & \multicolumn{1}{l|}{45.22}                                          & \multicolumn{1}{l|}{-}          & \multicolumn{1}{l|}{-}          & \multicolumn{1}{l|}{-}          & 21'950                          \\ \hline
		1/50                                                  & \multicolumn{1}{l|}{4'605}                                          & \multicolumn{1}{l|}{-}          & \multicolumn{1}{l|}{-}          & \multicolumn{1}{l|}{-}          & -                               \\ \hline
	\end{tabular}
\end{table}

На рисунке \ref{img:viz} представлена визуализация данных из таблицы \ref{tbl:time}.·

\begin{figure}[ht]
	\begin{center}
		\begin{tikzpicture}
			\begin{axis}[
				xlabel={$N$},
				ylabel={Время, мс},
				xmin=5, xmax=55,
				ymin=0, ymax=12000,
				legend pos=north west,
				ymajorgrids=true,
				grid style=dashed,
				]

				\addplot[color=blue, mark=oplus*]
				coordinates {
					(11, 0.17) (21, 0.52) (31, 1.08) (41, 1.80) (51, 2.74)
				};
				\addplot[color=red, mark=square]
				coordinates {
					(11, 220) (21, 2990) (31, 14615) (41, 45529) %(51, 110545)
				};
				\addplot[color=black, mark=square*]
				coordinates {
					(11, 101) (21, 769) (31, 2515) (41, 5793) (51, 11111)
				};
				\addplot[color=brown, mark=o]
				coordinates {
					(11, 71.56) (21, 379) (31, 987) (41, 1937) (51, 3257)
				};
				\addplot[color=magenta, mark=triangle*]
				coordinates {
					(11, 45.59) (21, 201) (31, 466) (41, 837) (51, 1321)
				};
				\legend{ПП, В1, В2, В3, В4}
			\end{axis}
		\end{tikzpicture}
		\captionsetup{justification=centering}
		\caption{Зависимость времени работы методов от $N$}
		\label{img:viz}
	\end{center}
\end{figure}

Таким образом, по результатам моделирования видно, что алгоритм, основанный на конечно-разностной аппроксимации дифференциального уравнения \eqref{eqn:source}, является более быстрым по времени по сравнению с вероятностным методом. Предложенные улучшения стохастического метода уменьшили время моделирования более, чем в $4N$ раз.

\section*{Заключение}

В ходе численного решения стационарного многомерного эллиптического уравнения теплопроводности выведены математические соотношения конечно-разностной аппроксимации (коэффициенты для метода матричной прогонки решения продольно-поперечной схемы) и вероятностного метода. Проведён анализ программных реализаций методов на быстродействие. Выяснено, что наибольший интерес в плане производительности представляет конечно-разностная аппроксимация. Был предложен ряд улучшений вероятностного метода, уменьшивший время моделирования более, чем в $4N$ раз, т. е. асимптотическая сложность алгоритма понижена на порядок.

\begin{thebibliography}{5}
	\bibitem{1} Градов В. М. Курс лекций по моделированию. МГТУ им. Н. Э. Баумана. — 2020
	\bibitem{2} Кузнецов В. Ф. Решение задач теплопроводности методом Монте-Карло. — 1973
\end{thebibliography}

\noindent \textbf{Керимов Ахмед Шахович} — студент, МГТУ им. Н. Э. Баумана, кафедра «Программное обеспечение ЭВМ и информационные технологии».


\end{document}
