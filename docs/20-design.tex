%% Методические указания к выполнению, оформлению и защите выпускной квалификационной работы бакалавра
%% 2.5 Конструкторский раздел
%%
%% В конструкторском разделе описывается разрабатываемый и/или модифицируемый метод или алгоритм.
%%
%% В случае если в бакалаврском проекте разрабатывается новый метод или алгоритм, необходимо подробно изложить их суть, привести всё необходимые для их реализации математические выкладки, обосновать последовательность этапов выполнения.
%% При этом для каждого этапа следует выделить необходимые исходные данные и получаемые результаты.
%%
%% При использовании известного алгоритма следует указать специфические особенности его практической реализации, присущие решаемой задаче, и пути их решения в ходе программирования.
%% Для описания метода или алгоритма необходимо выбрать наиболее подходящую форму записи (схема (ГОСТ 19.701-90), диаграмма деятельности, псевдокод и т. п.).
%% Учитывая, что на эффективность алгоритма непосредственно влияют используемые структуры данных, в данном разделе РПЗ целесообразно провести сравнительный анализ структур, которые могут быть применены в рамках программной реализации выбранного алгоритма, и обосновать выбор одной из них.
%% В конце описания разработанного и/или модифицируемого алгоритма должны быть приведены выбранные способы тестирования и сами тесты.
%%
%% Перед формированием тестовых наборов данных целесообразно указать выделенные классы эквивалентности.
%% В данной части расчётно-пояснительной записки могут также выполняться расчёты для определения объёмов памяти, необходимой для хранения данных, промежуточных и окончательных результатов работы программы, а также расчёты, позволяющие оценить время решения задачи на ЭВМ.
%% Эти результаты могут использоваться для обоснования правильности выбора метода и/или алгоритма из имеющихся альтернативных вариантов, а также для оценки возможности практически реализовать поставленную задачу на имеющейся технической базе.
%%
%% Другой важный момент, который должен найти своё отражение в конструкторском разделе, это описание структуры разрабатываемого программного обеспечения.
%% Обычно оно включает в себя:
%% — описание общей структуры — определение основных частей (компонентов) и их взаимосвязей по управлению и по данным;
%% — декомпозицию компонентов и построение структурных иерархий;
%% — проектирование компонентов.
%%
%% Для графического представления такого описания, если есть необходимость, следует использовать:
%% — функциональную модель IDEF0 с декомпозицией решения исходной задачи на несколько уровней (разрабатываемые модули обычно играют роль механизмов);
%% — спецификации компонентов (процессов);
%% — модель данных (ER-диаграмма);
%% — диаграмму классов;
%% — диаграмму компонентов;
%% — диаграмму переходов состояний (конечный автомат), характеризующих поведение системы во времени.
%%
%% Рекомендуемый объем конструкторского раздела 25—30 страниц.


\chapter{Конструкторский раздел}

\section{Продольно-поперечная схема}

%Обозначим поток тепла за
%
%\begin{equation}
%	F = -k(u) \frac{\partial u}{\partial x}.
%\end{equation}
%
%Тогда уравнение примет вид
%
%\begin{equation}
%	- \frac{\partial F}{\partial x} - \frac{\partial F}{\partial z} + f(x, z) = 0.
%\end{equation}

Добавим в эллиптическое уравнение координату по времени, чтобы получилось уравнение параболического типа с двумя пространственными переменными:

\begin{equation}
	\frac{\partial u}{\partial t} = \frac{\partial^2 u}{\partial x^2} + \frac{\partial^2 u}{\partial z^2} + \frac{f(x, z)}{k}.
\end{equation}

Для составления схемы, которая носит название продольно-поперченой, введём полуцелый слой $\overline{t} = t_m + \frac\tau2$.
Схема имеет вид
\begin{equation}
	\label{eqn:schema1}
	\frac{\overline y_{ij} - y_{ij}}{0,5\tau} = A_1\overline y_{ij} + A_2y_{ij} + \frac{f_{ij}}k,
\end{equation}
\begin{equation}
	\label{eqn:schema2}
	\frac{\hat y_{ij} - \overline y_{ij}}{0,5\tau} = A_1\overline y_{ij} + A_2\hat y_{ij} + \frac{f_{ij}}k,
\end{equation}
причём разностные операторы $A_1$, $A_2$ действуют каждый по своему направлению (по своей координате) и определяются выражениями
\begin{equation}
	A_1 y_{ij} = \frac{1}{h_x^2} (y_{i - 1, j} - 2y_{ij} + y_{i + 1, j}),
\end{equation}
\begin{equation}
	A_2 y_{ij} = \frac{1}{h_z^2} (y_{i, j - 1} - 2y_{ij} + y_{i, j + 1}).
\end{equation}
Здесь $1 \leqslant i \leqslant N_x - 1$, $1 \leqslant j \leqslant N_z - 1$.

Схема \eqref{eqn:schema1}, \eqref{eqn:schema2} реализуется следующим образом.
Вначале вычисляют решение на полуцелом слое согласно \eqref{eqn:schema1}.
В системе линейных уравнений \eqref{eqn:schema1} с трёхдиагональной матрицей неизвестными являются величины $\overline y_{ij}$, которые находят прогонкой по индексу $i$ (по координате $x$) для каждого фиксированного значения индекса $j$.
При найденном решении $\overline y_{ij}$ система \eqref{eqn:schema2} также является линейной системой уравнений с трёхдиагональной матрицей, в которой неизвестными выступают $\hat y_{ij}$.
Решение $\hat y_{ij}$ находят прогонкой по индексу $j$ (по координате $z$) для каждого индекса $i$.

Относительно аппроксимации и устойчивости продольно-поперечной схемы следует отметить, что схема \eqref{eqn:schema1}, \eqref{eqn:schema2} равномерно и безусловно устойчива по начальным данным и по правой части и аппроксимирует задачу на равномерных сетках с погрешностью $O(\tau^2 + h_x^2 + h_z^2)$.

Приведём уравнение \eqref{eqn:schema1} к стандартному виду для прогонки
\begin{equation}
	\frac{2(\overline y_{ij} - y_{ij})}{\tau} = \frac{1}{h_x^2} (\overline y_{i - 1, j} - 2\overline y_{ij} + \overline y_{i + 1, j}) + \frac{1}{h_z^2} (y_{i, j - 1} - 2y_{ij} + y_{i, j + 1}) + \frac{f_{ij}}k,
\end{equation}
\begin{multline*}
	\frac{1}{h_x^2}\overline y_{i - 1, j} - 2 \left(\frac1{h_x^2} + \frac1\tau \right)\overline y_{ij} + \frac{1}{h_x^2}\overline y_{i + 1, j} = \\
	= -\frac{1}{h_z^2}y_{i, j - 1} + 2\left(\frac1{h_z^2} - \frac1\tau\right)y_{ij} - \frac{1}{h_z^2}y_{i, j + 1} - \frac{f_{ij}}k,
\end{multline*}
\begin{multline*}
	\tau h_z^2 \overline y_{i - 1, j} - 2h_z^2(\tau + h_x^2)\overline y_{ij} + \tau h_z^2 \overline y_{i + 1, j} = \\
	= h_x^2 \left( 2(\tau - h_z^2)y_{ij} - \tau \left( y_{i, j - 1} + y_{i, j + 1} + \frac{h_z^2 f_{ij}}k \right) \right)
\end{multline*}

Коэффициенты для метода прогонки:
\begin{equation}
	\begin{dcases}
		A_i = \tau h_z^2, \\
		B_i = -2h_z^2(\tau + h_x^2), \\
		C_i = \tau h_z^2, \\
		D_i = h_x^2 \left( 2(\tau - h_z^2)y_{ij} - \tau \left( y_{i, j - 1} + y_{i, j + 1} + \frac{h_z^2 f_{ij}}k \right) \right), \\
		i = \overline{1, N_x - 1}. \\
	\end{dcases}
\end{equation}

Краевые условия:
\begin{equation}
	\begin{aligned}
		A_0     &= 0, & B_0     &= 1, & C_0     &= 0, & F_0     &= u_0, \\
		A_{N_x} &= 0, & B_{N_x} &= 1, & C_{N_x} &= 0, & F_{N_x} &= u_0. \\
	\end{aligned}
\end{equation}


Приведём уравнение \eqref{eqn:schema2} к стандартному виду для прогонки
\begin{equation}
	\frac{2(\hat y_{ij} - \overline y_{ij})}{\tau} = \frac{1}{h_x^2} (\overline y_{i - 1, j} - 2\overline y_{ij} + \overline y_{i + 1, j}) + \frac{1}{h_z^2} (\hat y_{i, j - 1} - 2\hat y_{ij} + \hat y_{i, j + 1}) + \frac{f_{ij}}k,
\end{equation}
\begin{multline*}
	\frac1{h_z^2} \hat y_{i, j - 1} - 2 \left(\frac1{h_z^2} + \frac1\tau \right)\hat y_{ij} + \frac1{h_z^2}\hat y_{i, j + 1} = \\
	= -\frac1{h_x^2}\overline y_{i - 1, j} + 2\left(\frac1{h_x^2} - \frac1\tau\right)\overline y_{ij} - \frac1{h_x^2}y_{i + 1, j} - \frac{f_{ij}}k,
\end{multline*}
\begin{multline*}
	\tau h_x^2 \hat y_{i, j - 1} - 2h_x^2(\tau + h_z^2)\hat y_{ij} + \tau h_x^2 \hat y_{i, j + 1} = \\
	= h_z^2 \left( 2(\tau - h_x^2)\overline y_{ij} - \tau \left( \overline y_{i - 1, j} + \overline y_{i + 1, j} + \frac{h_x^2 f_{ij}}k \right) \right)
\end{multline*}

Коэффициенты для метода прогонки:
\begin{equation}
	\begin{dcases}
		A_j = \tau h_x^2, \\
		B_j = -2h_x^2(\tau + h_z^2), \\
		C_j = \tau h_x^2, \\
		D_j = h_z^2 \left( 2(\tau - h_x^2)\overline y_{ij} - \tau \left( \overline y_{i - 1, j} + \overline y_{i + 1, j} + \frac{h_x^2 f_{ij}}k \right) \right), \\
		j = \overline{1, N_z - 1}. \\
	\end{dcases}
\end{equation}

Краевые условия:
\begin{equation}
	\begin{aligned}
		A_0     &= 0, & B_0     &= 1, & C_0     &= 0, & F_0     &= u_0, \\
		A_{N_z} &= 0, & B_{N_z} &= 1, & C_{N_z} &= 0, & F_{N_z} &= u_0. \\
	\end{aligned}
\end{equation}


\section{Выводы}
